%% Generated by Sphinx.
\def\sphinxdocclass{report}
\documentclass[letterpaper,10pt,english]{sphinxmanual}
\ifdefined\pdfpxdimen
   \let\sphinxpxdimen\pdfpxdimen\else\newdimen\sphinxpxdimen
\fi \sphinxpxdimen=.75bp\relax
\ifdefined\pdfimageresolution
    \pdfimageresolution= \numexpr \dimexpr1in\relax/\sphinxpxdimen\relax
\fi
%% let collapsible pdf bookmarks panel have high depth per default
\PassOptionsToPackage{bookmarksdepth=5}{hyperref}

\PassOptionsToPackage{warn}{textcomp}
\usepackage[utf8]{inputenc}
\ifdefined\DeclareUnicodeCharacter
% support both utf8 and utf8x syntaxes
  \ifdefined\DeclareUnicodeCharacterAsOptional
    \def\sphinxDUC#1{\DeclareUnicodeCharacter{"#1}}
  \else
    \let\sphinxDUC\DeclareUnicodeCharacter
  \fi
  \sphinxDUC{00A0}{\nobreakspace}
  \sphinxDUC{2500}{\sphinxunichar{2500}}
  \sphinxDUC{2502}{\sphinxunichar{2502}}
  \sphinxDUC{2514}{\sphinxunichar{2514}}
  \sphinxDUC{251C}{\sphinxunichar{251C}}
  \sphinxDUC{2572}{\textbackslash}
\fi
\usepackage{cmap}
\usepackage[T1]{fontenc}
\usepackage{amsmath,amssymb,amstext}
\usepackage{babel}



\usepackage{tgtermes}
\usepackage{tgheros}
\renewcommand{\ttdefault}{txtt}



\usepackage[Bjarne]{fncychap}
\usepackage{sphinx}

\fvset{fontsize=auto}
\usepackage{geometry}


% Include hyperref last.
\usepackage{hyperref}
% Fix anchor placement for figures with captions.
\usepackage{hypcap}% it must be loaded after hyperref.
% Set up styles of URL: it should be placed after hyperref.
\urlstyle{same}

\addto\captionsenglish{\renewcommand{\contentsname}{Contents:}}

\usepackage{sphinxmessages}
\setcounter{tocdepth}{1}



\title{MAD\sphinxhyphen{}NG}
\date{Aug 30, 2022}
\release{0.9.6}
\author{Laurent Deniau}
\newcommand{\sphinxlogo}{\vbox{}}
\renewcommand{\releasename}{Release}
\makeindex
\begin{document}

\ifdefined\shorthandoff
  \ifnum\catcode`\=\string=\active\shorthandoff{=}\fi
  \ifnum\catcode`\"=\active\shorthandoff{"}\fi
\fi

\pagestyle{empty}
\sphinxmaketitle
\pagestyle{plain}
\sphinxtableofcontents
\pagestyle{normal}
\phantomsection\label{\detokenize{index::doc}}


\sphinxstepscope


\chapter{Sequences}
\label{\detokenize{sequences:sequences}}\label{\detokenize{sequences::doc}}
\sphinxAtStartPar
The MAD Sequences are objects convenient to describe accelerators lattices built from a \sphinxstyleemphasis{list} of elements with increasing \sphinxcode{\sphinxupquote{s}}\sphinxhyphen{}positions. The sequences are also containers that provide fast access to their elements by referring to their indexes, \sphinxcode{\sphinxupquote{s}}\sphinxhyphen{}positions, or (mangled) names, or by running iterators constrained with ranges and predicates.
The \sphinxcode{\sphinxupquote{sequence}} object is the \sphinxstyleemphasis{root object} of sequences that store information relative to lattices.

\sphinxAtStartPar
The \sphinxcode{\sphinxupquote{sequence}} module extends the {\hyperref[\detokenize{types::doc}]{\sphinxcrossref{\DUrole{doc}{typeid}}}} module with the \sphinxcode{\sphinxupquote{is\_sequence}} function, which returns \sphinxcode{\sphinxupquote{true}} if its argument is a \sphinxcode{\sphinxupquote{sequence}} object, \sphinxcode{\sphinxupquote{false}} otherwise.


\section{Attributes}
\label{\detokenize{sequences:attributes}}
\sphinxAtStartPar
The \sphinxcode{\sphinxupquote{sequence}} object provides the following attributes:
\begin{description}
\sphinxlineitem{\sphinxstylestrong{l}}
\sphinxAtStartPar
A \sphinxstyleemphasis{number} specifying the length of the sequence \sphinxcode{\sphinxupquote{{[}m{]}}}. A \sphinxcode{\sphinxupquote{nil}} will be replaced by the computed lattice length. A value greater or equal to the computed lattice length will be used to place the \sphinxcode{\sphinxupquote{\$end}} marker. Other values will raise an error. (default: \sphinxcode{\sphinxupquote{nil}}).

\sphinxlineitem{\sphinxstylestrong{dir}}
\sphinxAtStartPar
A \sphinxstyleemphasis{number} holding one of \sphinxcode{\sphinxupquote{1}} (forward) or \sphinxcode{\sphinxupquote{\sphinxhyphen{}1}} (backward) and specifying the direction of the sequence. %
\begin{footnote}[1]\sphinxAtStartFootnote
This is equivalent to the MAD\sphinxhyphen{}X \sphinxcode{\sphinxupquote{bv}} flag.
%
\end{footnote} (default:\textasciitilde{} \sphinxcode{\sphinxupquote{1}})

\sphinxlineitem{\sphinxstylestrong{refer}}
\sphinxAtStartPar
A \sphinxstyleemphasis{string} holding one of \sphinxcode{\sphinxupquote{"entry"}}, \sphinxcode{\sphinxupquote{"centre"}} or    return true \sphinxcode{\sphinxupquote{"exit"}} to specify the default reference position in the elements to use for their placement. An element can override it with its \sphinxcode{\sphinxupquote{refpos}} attribute, see {\hyperref[\detokenize{sequences:element-positions}]{\sphinxcrossref{element positions}}} details. (default: \sphinxcode{\sphinxupquote{nil}} \(\equiv\) \sphinxcode{\sphinxupquote{"centre"}}).

\sphinxlineitem{\sphinxstylestrong{owner}}
\sphinxAtStartPar
A \sphinxstyleemphasis{logical} specifying if an \sphinxstyleemphasis{empty} sequence is a view with no data \sphinxcode{\sphinxupquote{(owner \textasciitilde{}= true)}}, or a sequence holding data \sphinxcode{\sphinxupquote{(owner == true)}}. (default: \sphinxcode{\sphinxupquote{nil}})

\sphinxlineitem{\sphinxstylestrong{minlen}}
\sphinxAtStartPar
A \sphinxstyleemphasis{number} specifying the minimal length \sphinxcode{\sphinxupquote{{[}m{]}}} to generate \sphinxstyleemphasis{implicit} drifts between elements in \(s\)\sphinxhyphen{}iterators generated by the method \sphinxcode{\sphinxupquote{:siter}}. This attribute is automatically set to \(10^{-6}\) m when a sequence is created within the MADX environment. (default: \sphinxcode{\sphinxupquote{nil}})

\sphinxlineitem{\sphinxstylestrong{beam}}
\sphinxAtStartPar
An attached \sphinxcode{\sphinxupquote{beam}}. (default: \sphinxcode{\sphinxupquote{nil}})

\end{description}

\sphinxAtStartPar
\sphinxstylestrong{Warning}: the following private and read\sphinxhyphen{}only attributes are present in all sequences and should \sphinxstyleemphasis{never be used, set or changed}; breaking this rule would lead to an \sphinxstyleemphasis{undefined behavior}:
\begin{description}
\sphinxlineitem{\sphinxstylestrong{\_\_dat}}
\sphinxAtStartPar
A \sphinxstyleemphasis{table} containing all the private data of sequences.

\sphinxlineitem{\sphinxstylestrong{\_\_cycle}}
\sphinxAtStartPar
A \sphinxstyleemphasis{reference} to the element registered with the \sphinxcode{\sphinxupquote{:cycle}} method. (default: \sphinxcode{\sphinxupquote{nil}})

\end{description}


\section{Methods}
\label{\detokenize{sequences:methods}}
\sphinxAtStartPar
The \sphinxcode{\sphinxupquote{sequence}} object provides the following methods:
\begin{description}
\sphinxlineitem{\sphinxstylestrong{elem}}
\sphinxAtStartPar
A \sphinxstyleemphasis{method} \sphinxcode{\sphinxupquote{(idx)}} returning the element stored at the positive index \sphinxcode{\sphinxupquote{idx}} in the sequence, or \sphinxcode{\sphinxupquote{nil}}.

\sphinxlineitem{\sphinxstylestrong{spos}}
\sphinxAtStartPar
A \sphinxstyleemphasis{method} \sphinxcode{\sphinxupquote{(idx)}} returning the \(s\)\sphinxhyphen{}position at the entry of the element stored at the positive index \sphinxcode{\sphinxupquote{idx}} in the sequence, or \sphinxcode{\sphinxupquote{nil}}.

\sphinxlineitem{\sphinxstylestrong{upos}}
\sphinxAtStartPar
A \sphinxstyleemphasis{method} \sphinxcode{\sphinxupquote{(idx)}} returning the \(s\)\sphinxhyphen{}position at the user\sphinxhyphen{}defined \sphinxcode{\sphinxupquote{refpos}} offset of the element stored at the positive index \sphinxcode{\sphinxupquote{idx}}
in the sequence, or \sphinxcode{\sphinxupquote{nil}}.

\sphinxlineitem{\sphinxstylestrong{ds}}
\sphinxAtStartPar
A \sphinxstyleemphasis{method} \sphinxcode{\sphinxupquote{(idx)}} returning the length of the element stored at the positive index \sphinxcode{\sphinxupquote{idx}} in the sequence, or \sphinxcode{\sphinxupquote{nil}}.

\sphinxlineitem{\sphinxstylestrong{align}}
\sphinxAtStartPar
A \sphinxstyleemphasis{method} \sphinxcode{\sphinxupquote{(idx)}} returning a \sphinxstyleemphasis{set} specifying the misalignment of the element stored at the positive index \sphinxcode{\sphinxupquote{idx}} in the sequence, or \sphinxcode{\sphinxupquote{nil}}.

\sphinxlineitem{\sphinxstylestrong{index}}
\sphinxAtStartPar
A \sphinxstyleemphasis{method} \sphinxcode{\sphinxupquote{(idx)}} returning a positive index, or \sphinxcode{\sphinxupquote{nil}}. If \sphinxcode{\sphinxupquote{idx}} is negative, it is reflected versus the size of the sequence, e.g. \sphinxcode{\sphinxupquote{\sphinxhyphen{}1}}
becomes \sphinxcode{\sphinxupquote{\#self}}, the index of the \sphinxcode{\sphinxupquote{\$end}} marker.

\sphinxlineitem{\sphinxstylestrong{name\_of}}
\sphinxAtStartPar
A \sphinxstyleemphasis{method} \sphinxcode{\sphinxupquote{(idx, {[}ref{]})}} returning a \sphinxstyleemphasis{string} corresponding to the (mangled) name of the element at the index \sphinxcode{\sphinxupquote{idx}} or \sphinxcode{\sphinxupquote{nil}}. An element
name appearing more than once in the sequence will be mangled with an absolute count, e.g. \sphinxcode{\sphinxupquote{mq{[}3{]}}}, or a relative count versus the optional
reference element \sphinxcode{\sphinxupquote{ref}} determined by \sphinxcode{\sphinxupquote{:index\_of}}, e.g. \sphinxcode{\sphinxupquote{mq\{\sphinxhyphen{}2\}}}.

\sphinxlineitem{\sphinxstylestrong{index\_of}}
\sphinxAtStartPar
A \sphinxstyleemphasis{method} \sphinxcode{\sphinxupquote{(a, {[}ref{]}, {[}dir{]})}} returning a \sphinxstyleemphasis{number} corresponding to the positive index of the element determined by the first argument or \sphinxcode{\sphinxupquote{nil}}.
If \sphinxcode{\sphinxupquote{a}} is a \sphinxstyleemphasis{number} (or a \sphinxstyleemphasis{string} representing a \sphinxstyleemphasis{number}), it is interpreted as the \(s\)\sphinxhyphen{}position of an element and returned as a second
\sphinxstyleemphasis{number}. If \sphinxcode{\sphinxupquote{a}} is a \sphinxstyleemphasis{string}, it is interpreted as the (mangled) name of an element as returned by \sphinxcode{\sphinxupquote{:name\_of}}. Finally, \sphinxcode{\sphinxupquote{a}} can be a \sphinxstyleemphasis{reference}
to an element to search for. The argument \sphinxcode{\sphinxupquote{ref}} (default: \sphinxcode{\sphinxupquote{nil)}} specifies the reference element determined by \sphinxcode{\sphinxupquote{:index\_of(ref)}} to use for
relative \(s\)\sphinxhyphen{}positions, for decoding mangled names with relative counts, or as the element to start searching from. The argument \sphinxcode{\sphinxupquote{dir}}
(default: \sphinxcode{\sphinxupquote{1)}} specifies the direction of the search with values \sphinxcode{\sphinxupquote{1}} (forward), \sphinxcode{\sphinxupquote{\sphinxhyphen{}1}} (backward), or \sphinxcode{\sphinxupquote{0}} (no direction). The \sphinxcode{\sphinxupquote{dir=0}}
case may return an index at half\sphinxhyphen{}integer if \sphinxcode{\sphinxupquote{a}} is interpreted as an \(s\)\sphinxhyphen{}position pointing to an \sphinxstyleemphasis{implicit drift}.

\sphinxlineitem{\sphinxstylestrong{range\_of}}
\sphinxAtStartPar
A \sphinxstyleemphasis{method} \sphinxcode{\sphinxupquote{({[}rng{]}, {[}ref{]}, {[}dir{]}}}) returning three \sphinxstyleemphasis{numbers} corresponding to the positive indexes \sphinxstyleemphasis{start} and \sphinxstyleemphasis{end} of the range and
its direction \sphinxstyleemphasis{dir}, or \sphinxcode{\sphinxupquote{nil}} for an empty range. If \sphinxcode{\sphinxupquote{rng}} is omitted, it returns \sphinxcode{\sphinxupquote{1}}, \sphinxcode{\sphinxupquote{\#self}}, \sphinxcode{\sphinxupquote{1}}, or \sphinxcode{\sphinxupquote{\#self}}, \sphinxcode{\sphinxupquote{1}}, \sphinxcode{\sphinxupquote{\sphinxhyphen{}1}}
if \sphinxcode{\sphinxupquote{dir}} is negative. If \sphinxcode{\sphinxupquote{rng}} is a \sphinxstyleemphasis{number} or a \sphinxstyleemphasis{string} with no \sphinxcode{\sphinxupquote{\textquotesingle{}/\textquotesingle{}}} separator, it is interpreted as both \sphinxstyleemphasis{start} and \sphinxstyleemphasis{end} and
determined by \sphinxcode{\sphinxupquote{index\_of}}. If \sphinxcode{\sphinxupquote{rng}} is a \sphinxstyleemphasis{string} containing the separator \sphinxcode{\sphinxupquote{\textquotesingle{}/\textquotesingle{}}}, it is split in two \sphinxstyleemphasis{strings} interpreted as \sphinxstyleemphasis{start}
and \sphinxstyleemphasis{end}, both determined by \sphinxcode{\sphinxupquote{:index\_of}}. If \sphinxcode{\sphinxupquote{rng}} is a \sphinxstyleemphasis{list}, it will be interpreted as \{\sphinxstyleemphasis{start}, \sphinxstyleemphasis{end}, \sphinxcode{\sphinxupquote{{[}ref{]}}}, \sphinxcode{\sphinxupquote{{[}dir{]}}}\},
both determined by \sphinxcode{\sphinxupquote{:index\_of}}, unless \sphinxcode{\sphinxupquote{ref}} equals \sphinxcode{\sphinxupquote{\textquotesingle{}idx\textquotesingle{}}} then both are determined by \sphinxcode{\sphinxupquote{:index}} (i.e. a \sphinxstyleemphasis{number} is interpreted as an
index instead of a \(s\)\sphinxhyphen{}position). The arguments \sphinxcode{\sphinxupquote{ref}} (default: \sphinxcode{\sphinxupquote{nil}}) and \sphinxcode{\sphinxupquote{dir}} (default: \sphinxcode{\sphinxupquote{1}}) are forwarded to all invocations
of \sphinxcode{\sphinxupquote{:index\_of}} with a higher precedence than ones in the \sphinxstyleemphasis{list} \sphinxcode{\sphinxupquote{rng}}, and a runtime error is raised if the method returns \sphinxcode{\sphinxupquote{nil}}, i.e.
to disambiguate between a valid empty range and an invalid range.

\sphinxlineitem{\sphinxstylestrong{length\_of}}
\sphinxAtStartPar
A \sphinxstyleemphasis{method} \sphinxcode{\sphinxupquote{({[}rng{]}, {[}ntrn{]}, {[}dir{]}}}) returning a \sphinxstyleemphasis{number} specifying the length of the range optionally including \sphinxcode{\sphinxupquote{ntrn}} extra turns (default: \sphinxcode{\sphinxupquote{0}}),
and calculated from the indexes returned by \sphinxcode{\sphinxupquote{:range\_of({[}rng{]}, nil, {[}dir{]})}}.

\sphinxlineitem{\sphinxstylestrong{iter}}
\sphinxAtStartPar
A \sphinxstyleemphasis{method} \sphinxcode{\sphinxupquote{({[}rng{]}, {[}ntrn{]}, {[}dir{]})}} returning an iterator over the sequence elements. The optional range is determined by
TT\{:range\_of(rng, {[}dir{]})\}, optionally including \sphinxcode{\sphinxupquote{ntrn}} turns (default: \sphinxcode{\sphinxupquote{0}}). The optional direction \sphinxcode{\sphinxupquote{dir}} specifies the forward \sphinxcode{\sphinxupquote{1}}
or the backward \sphinxcode{\sphinxupquote{\sphinxhyphen{}1}} direction of the iterator. If \sphinxcode{\sphinxupquote{rng}} is not provided and the ?sequence? is cycled, the \sphinxstyleemphasis{start} and \sphinxstyleemphasis{end} indexes are
determined by \sphinxcode{\sphinxupquote{:index\_of(self.\_\_cycle)}}. When used with a generic \sphinxcode{\sphinxupquote{for}} loop, the iterator returns at each element: its index,
the element itself, its \(s\)\sphinxhyphen{}position over the running loop and its signed length depending on the direction.

\sphinxlineitem{\sphinxstylestrong{siter}}
\sphinxAtStartPar
A \sphinxstyleemphasis{method} \sphinxcode{\sphinxupquote{({[}rng{]}, {[}ntrn{]}, {[}dir{]})}} returning an \(s\)\sphinxhyphen{}iterator over the sequence elements. The optional range is determined by
\sphinxcode{\sphinxupquote{:range\_of({[}rng{]}, nil, {[}dir{]})}}, optionally including \sphinxcode{\sphinxupquote{ntrn}} turns (default: \sphinxcode{\sphinxupquote{0}}). The optional direction \sphinxcode{\sphinxupquote{dir}} specifies the
forward \sphinxcode{\sphinxupquote{1}} or the backward \sphinxcode{\sphinxupquote{\sphinxhyphen{}1}} direction of the iterator. When used with a generic \sphinxcode{\sphinxupquote{for}} loop, the iterator returns at each
iteration: its index, the element itself or an \sphinxstyleemphasis{implicit} \sphinxcode{\sphinxupquote{drift}}, its \(s\)\sphinxhyphen{}position over the running loop and its signed length
depending on the direction. Each \sphinxstyleemphasis{implicit} drift is built on\sphinxhyphen{}the\sphinxhyphen{}fly by the iterator with a length equal to the gap between the elements
surrounding it and a half\sphinxhyphen{}integer index equal to the average of their indexes. The length of \sphinxstyleemphasis{implicit} drifts is bounded by the maximum
between the sequence attribute \sphinxcode{\sphinxupquote{minlen}} and the \sphinxcode{\sphinxupquote{minlen}} from the {\hyperref[\detokenize{constants::doc}]{\sphinxcrossref{\DUrole{doc}{constant}}}} module.

\sphinxlineitem{\sphinxstylestrong{foreach}}
\sphinxAtStartPar
A \sphinxstyleemphasis{method} \sphinxcode{\sphinxupquote{(act, {[}rng{]}, {[}sel{]}, {[}not{]})}} returning the sequence itself after applying the action \sphinxcode{\sphinxupquote{act}} on the selected elements. If \sphinxcode{\sphinxupquote{act}}
is a \sphinxstyleemphasis{set} representing the arguments in the packed form, the missing arguments will be extracted from the attributes \sphinxcode{\sphinxupquote{action}},
\sphinxcode{\sphinxupquote{range}}, \sphinxcode{\sphinxupquote{select}} and \sphinxcode{\sphinxupquote{default}}. The action \sphinxcode{\sphinxupquote{act}} must be a \sphinxstyleemphasis{callable} \sphinxcode{\sphinxupquote{(elm, idx, {[}midx{]})}} applied to an element passed as
first argument and its index as second argument, the optional third argument being the index of the main element in case \sphinxcode{\sphinxupquote{elm}} is a sub\sphinxhyphen{}element.
The optional range is used to generate the loop iterator \sphinxcode{\sphinxupquote{:iter({[}rng{]})}}. The optional selector \sphinxcode{\sphinxupquote{sel}} is a \sphinxstyleemphasis{callable} \sphinxcode{\sphinxupquote{(elm, idx, {[}midx{]})}}
predicate selecting eligible elements for the action using the same arguments. The selector \sphinxcode{\sphinxupquote{sel}} can be specified in other ways,
see {\hyperref[\detokenize{sequences:element-selections}]{\sphinxcrossref{element selections}}} for details. The optional \sphinxstyleemphasis{logical} \sphinxcode{\sphinxupquote{not}} (default: \sphinxcode{\sphinxupquote{false}}) indicates how to interpret default selection, as
\sphinxstyleemphasis{all} or \sphinxstyleemphasis{none}, depending on the semantic of the action. %
\begin{footnote}[2]\sphinxAtStartFootnote
For example, the \sphinxcode{\sphinxupquote{:remove}} method needs \sphinxcode{\sphinxupquote{not=true}} to \sphinxstyleemphasis{not} remove all elements if no selector is provided.
%
\end{footnote}

\sphinxlineitem{\sphinxstylestrong{select}}
\sphinxAtStartPar
A \sphinxstyleemphasis{method} \sphinxcode{\sphinxupquote{({[}flg{]}, {[}rng{]}, {[}sel{]}, {[}not{]})}} returning the sequence itself after applying the action \sphinxcode{\sphinxupquote{:select({[}flg{]})}} to the elements using
\sphinxcode{\sphinxupquote{:foreach(act, {[}rng{]}, {[}sel{]}, {[}not{]})}}. By default sequence have all their elements deselected with only the \sphinxcode{\sphinxupquote{\$end}} marker \sphinxcode{\sphinxupquote{observed}}.

\sphinxlineitem{\sphinxstylestrong{deselect}}
\sphinxAtStartPar
A \sphinxstyleemphasis{method} \sphinxcode{\sphinxupquote{({[}flg{]}, {[}rng{]}, {[}sel{]}, {[}not{]})}} returning the sequence itself after applying the action \sphinxcode{\sphinxupquote{:deselect({[}flg{]})}} to the elements
using \sphinxcode{\sphinxupquote{:foreach(act, {[}rng{]}, {[}sel{]}, {[}not{]})}}. By default sequence have all their elements deselected with only the \sphinxcode{\sphinxupquote{\$end}} marker \sphinxcode{\sphinxupquote{observed}}.

\sphinxlineitem{\sphinxstylestrong{filter}}
\sphinxAtStartPar
A \sphinxstyleemphasis{method} \sphinxcode{\sphinxupquote{({[}rng{]}, {[}sel{]}, {[}not{]})}} returning a \sphinxstyleemphasis{list} containing the positive indexes of the elements determined by \sphinxcode{\sphinxupquote{:foreach(act, {[}rng{]}, {[}sel{]}, {[}not{]})}},
and its size. The \sphinxstyleemphasis{logical} \sphinxcode{\sphinxupquote{sel.subelem}} specifies to select sub\sphinxhyphen{}elements too, and the \sphinxstyleemphasis{list} may contain non\sphinxhyphen{}integer indexes encoding their main element
index added to their relative position, i.e. \sphinxcode{\sphinxupquote{midx.sat}}. The builtin \sphinxstyleemphasis{function} \sphinxcode{\sphinxupquote{math.modf(num)}} allows to retrieve easily the main element \sphinxcode{\sphinxupquote{midx}} and
the sub\sphinxhyphen{}element \sphinxcode{\sphinxupquote{sat}}, e.g. \sphinxcode{\sphinxupquote{midx,sat = math.modf(val)}}.

\sphinxlineitem{\sphinxstylestrong{install}}
\sphinxAtStartPar
A \sphinxstyleemphasis{method} \sphinxcode{\sphinxupquote{(elm, {[}rng{]}, {[}sel{]}, {[}cmp{]})}} returning the sequence itself after installing the elements in the \sphinxstyleemphasis{list} \sphinxcode{\sphinxupquote{elm}} at their
{\hyperref[\detokenize{sequences:element-positions}]{\sphinxcrossref{element positions}}}; unless \sphinxcode{\sphinxupquote{from="selected"}} is defined meaning multiple installations at positions relative to each element determined by the method
\sphinxcode{\sphinxupquote{:filter({[}rng{]}, {[}sel{]}, true)}}. The \sphinxstyleemphasis{logical} \sphinxcode{\sphinxupquote{sel.subelem}} is ignored. If the arguments are passed in the packed form, the extra attribute \sphinxcode{\sphinxupquote{elements}}
will be used as a replacement for the argument \sphinxcode{\sphinxupquote{elm}}. The \sphinxstyleemphasis{logical} \sphinxcode{\sphinxupquote{elm.subelem}} specifies to install elements with \(s\)\sphinxhyphen{}position falling inside
sequence elements as sub\sphinxhyphen{}elements, and set their \sphinxcode{\sphinxupquote{sat}} attribute accordingly. The optional \sphinxstyleemphasis{callable} \sphinxcode{\sphinxupquote{cmp(elmspos, spos{[}idx{]})}} (default: \sphinxcode{\sphinxupquote{"\textless{}"}}) is used
to search for the \(s\)\sphinxhyphen{}position of the installation, where equal \(s\)\sphinxhyphen{}position are installed after (i.e. before with \sphinxcode{\sphinxupquote{"\textless{}="}}), see \sphinxcode{\sphinxupquote{bsearch}} from
the {\hyperref[\detokenize{utility::doc}]{\sphinxcrossref{\DUrole{doc}{utility}}}} module for details. The \sphinxstyleemphasis{implicit} drifts are checked after each element installation.

\sphinxlineitem{\sphinxstylestrong{replace}}
\sphinxAtStartPar
A \sphinxstyleemphasis{method} \sphinxcode{\sphinxupquote{(elm, {[}rng{]}, {[}sel{]})}} returning the \sphinxstyleemphasis{list} of replaced elements by the elements in the \sphinxstyleemphasis{list} \sphinxcode{\sphinxupquote{elm}} placed at their {\hyperref[\detokenize{sequences:element-positions}]{\sphinxcrossref{element positions}}}, and the
\sphinxstyleemphasis{list} of their respective indexes, both determined by \sphinxcode{\sphinxupquote{:filter({[}rng{]}, {[}sel{]}, true)}}. The \sphinxstyleemphasis{list} \sphinxcode{\sphinxupquote{elm}} cannot contain instances of \sphinxcode{\sphinxupquote{sequence}} or \sphinxcode{\sphinxupquote{bline}}
elements and will be recycled as many times as needed to replace all selected elements. If the arguments are passed in the packed form, the extra attribute
\sphinxcode{\sphinxupquote{elements}} will be used as a replacement for the argument \sphinxcode{\sphinxupquote{elm}}. The \sphinxstyleemphasis{logical} \sphinxcode{\sphinxupquote{sel.subelem}} specifies to replace selected sub\sphinxhyphen{}elements too and set
their \sphinxcode{\sphinxupquote{sat}} attribute to the same value. The \sphinxstyleemphasis{implicit} drifts are checked only once all elements have been replaced.

\sphinxlineitem{\sphinxstylestrong{remove}}
\sphinxAtStartPar
A \sphinxstyleemphasis{method} \sphinxcode{\sphinxupquote{({[}rng{]}, {[}sel{]})}} returning the \sphinxstyleemphasis{list} of removed elements and the \sphinxstyleemphasis{list} of their respective indexes, both determined by \sphinxcode{\sphinxupquote{:filter({[}rng{]}, {[}sel{]}, true)}}.
The \sphinxstyleemphasis{logical} \sphinxcode{\sphinxupquote{sel.subelem}} specifies to remove selected sub\sphinxhyphen{}elements too.

\sphinxlineitem{\sphinxstylestrong{move}}
\sphinxAtStartPar
A \sphinxstyleemphasis{method} \sphinxcode{\sphinxupquote{({[}rng{]}, {[}sel{]})}} returning the sequence itself after updating the {\hyperref[\detokenize{sequences:element-positions}]{\sphinxcrossref{element positions}}} at the indexes determined by \sphinxcode{\sphinxupquote{:filter({[}rng{]}, {[}sel{]}, true)}}.
The \sphinxstyleemphasis{logical} \sphinxcode{\sphinxupquote{sel.subelem}} is ignored. The elements must keep their order in the sequence and surrounding \sphinxstyleemphasis{implicit} drifts are checked only once all elements
have been moved. %
\begin{footnote}[3]\sphinxAtStartFootnote
Updating directly the positions attributes of an element has no effect.
%
\end{footnote}

\sphinxlineitem{\sphinxstylestrong{misalign}}
\sphinxAtStartPar
A \sphinxstyleemphasis{method} \sphinxcode{\sphinxupquote{(algn, {[}rng{]}, {[}sel{]})}} returning the sequence itself after setting the {\hyperref[\detokenize{elements:elm-misalign}]{\sphinxcrossref{\DUrole{std,std-ref}{element misalignments}}}} from
\sphinxcode{\sphinxupquote{algn}} at the indexes determined by \sphinxcode{\sphinxupquote{:filter({[}rng{]}, {[}sel{]}, true)}}. If \sphinxcode{\sphinxupquote{algn}} is a \sphinxstyleemphasis{mappable}, it will be used to misalign the filtered elements.
If \sphinxcode{\sphinxupquote{algn}} is a \sphinxstyleemphasis{iterable}, it will be accessed using the filtered elements indexes to retrieve their specific misalignment.
If \sphinxcode{\sphinxupquote{algn}} is a \sphinxstyleemphasis{callable} \sphinxcode{\sphinxupquote{(idx)}}, it will be invoked for each filtered element with their index as solely argument to retrieve their specific misalignment.

\sphinxlineitem{\sphinxstylestrong{reflect}}
\sphinxAtStartPar
A \sphinxstyleemphasis{method} \sphinxcode{\sphinxupquote{({[}name{]})}} returning a new sequence from the sequence reversed, and named from the optional \sphinxstyleemphasis{string} \sphinxcode{\sphinxupquote{name}} (default: \sphinxcode{\sphinxupquote{self.name..\textquotesingle{}\_rev\textquotesingle{}}}).

\sphinxlineitem{\sphinxstylestrong{cycle}}
\sphinxAtStartPar
A \sphinxstyleemphasis{method} \sphinxcode{\sphinxupquote{(a)}} returning the sequence itself after checking that \sphinxcode{\sphinxupquote{a}} is a valid reference using \sphinxcode{\sphinxupquote{:index\_of(a)}}, and storing it in the \sphinxcode{\sphinxupquote{\_\_cycle}} attribute,
itself erased by the methods editing the sequence like \sphinxcode{\sphinxupquote{:install}}, \sphinxcode{\sphinxupquote{:replace}}, \sphinxcode{\sphinxupquote{:remove}}, \sphinxcode{\sphinxupquote{:share}}, and \sphinxcode{\sphinxupquote{:unique}}.

\sphinxlineitem{\sphinxstylestrong{share}}
\sphinxAtStartPar
A \sphinxstyleemphasis{method} \sphinxcode{\sphinxupquote{(seq2)}} returning the \sphinxstyleemphasis{list} of elements removed from the \sphinxcode{\sphinxupquote{seq2}} and the \sphinxstyleemphasis{list} of their respective indexes, and replaced by the elements from the
sequence with the same name when they are unique in both sequences.

\sphinxlineitem{\sphinxstylestrong{unique}}
\sphinxAtStartPar
A \sphinxstyleemphasis{method} \sphinxcode{\sphinxupquote{({[}fmt{]})}} returning the sequence itself after replacing all non\sphinxhyphen{}unique elements by new instances sharing the same parents.
The optional \sphinxcode{\sphinxupquote{fmt}} must be a \sphinxstyleemphasis{callable} \sphinxcode{\sphinxupquote{(name, cnt, idx)}} that returns the mangled name of the new instance build from the element \sphinxcode{\sphinxupquote{name}},
its count \sphinxcode{\sphinxupquote{cnt}} and its index \sphinxcode{\sphinxupquote{idx}} in the sequence. If the optional \sphinxcode{\sphinxupquote{fmt}} is a \sphinxstyleemphasis{string}, the mangling \sphinxstyleemphasis{callable} is built by binding \sphinxcode{\sphinxupquote{fmt}}
as first argument to the function \sphinxcode{\sphinxupquote{string.format}} from the standard library, see
\sphinxhref{http://github.com/MethodicalAcceleratorDesign/MADdocs/blob/master/lua52-refman-madng.pdf}{Lua 5.2} \S{}6.4 for details.

\sphinxlineitem{\sphinxstylestrong{publish}}
\sphinxAtStartPar
A \sphinxstyleemphasis{method} \sphinxcode{\sphinxupquote{(env, {[}keep{]})}} returning the sequence after publishing all its elements in the environment \sphinxcode{\sphinxupquote{env}}. If the \sphinxstyleemphasis{logical} \sphinxcode{\sphinxupquote{keep}} is
\sphinxcode{\sphinxupquote{true}}, the method will preserve existing elements from being overridden. This method is automatically invoked with \sphinxcode{\sphinxupquote{keep=true}} when sequences
are created within the \sphinxcode{\sphinxupquote{MADX}} environment.

\sphinxlineitem{\sphinxstylestrong{copy}}
\sphinxAtStartPar
A \sphinxstyleemphasis{method} \sphinxcode{\sphinxupquote{({[}name{]}, {[}owner{]})}} returning a new sequence from a copy of \sphinxcode{\sphinxupquote{self}}, with the optional \sphinxcode{\sphinxupquote{name}} and the optional attribute \sphinxcode{\sphinxupquote{owner}} set.
If the sequence is a view, so will be the copy unless \sphinxcode{\sphinxupquote{owner == true}}.

\sphinxlineitem{\sphinxstylestrong{is\_view}}
\sphinxAtStartPar
A \sphinxstyleemphasis{method} () returning \sphinxcode{\sphinxupquote{true}} if the sequence is a view over another sequence data, \sphinxcode{\sphinxupquote{false}} otherwise.

\sphinxlineitem{\sphinxstylestrong{set\_readonly}}
\sphinxAtStartPar
Set the sequence as read\sphinxhyphen{}only, including its columns.

\sphinxlineitem{\sphinxstylestrong{save\_flags}}
\sphinxAtStartPar
A \sphinxstyleemphasis{method} \sphinxcode{\sphinxupquote{({[}flgs{]})}} saving the flags of all the elements to the optional \sphinxstyleemphasis{iterable} \sphinxcode{\sphinxupquote{flgs}} (default: \sphinxcode{\sphinxupquote{\{\}}}) and return it.

\sphinxlineitem{\sphinxstylestrong{restore\_flags}}
\sphinxAtStartPar
A \sphinxstyleemphasis{method} \sphinxcode{\sphinxupquote{(flgs)}} restoring the flags of all the elements from the \sphinxstyleemphasis{iterable} \sphinxcode{\sphinxupquote{flgs}}. The indexes of the flags must match the indexes of the elements
in the sequence.

\sphinxlineitem{\sphinxstylestrong{dumpseq}}
\sphinxAtStartPar
A \sphinxstyleemphasis{method} \sphinxcode{\sphinxupquote{({[}fil{]}, {[}info{]})}}\} displaying on the optional file \sphinxcode{\sphinxupquote{fil}} (default: \sphinxcode{\sphinxupquote{io.stdout}}) information related to the position and length of the elements.
Useful to identify negative drifts and badly positioned elements. The optional argument \sphinxcode{\sphinxupquote{info}} indicates to display extra information like elements misalignments.

\sphinxlineitem{\sphinxstylestrong{check\_sequ}}
\sphinxAtStartPar
A \sphinxstyleemphasis{method} () checking the integrity of the sequence and its dictionary, for debugging purpose only.

\end{description}


\section{Metamethods}
\label{\detokenize{sequences:metamethods}}
\sphinxAtStartPar
The \sphinxcode{\sphinxupquote{sequence}} object provides the following metamethods:
\begin{description}
\sphinxlineitem{\sphinxstylestrong{\_\_len}}
\sphinxAtStartPar
A \sphinxstyleemphasis{method} () called by the length operator \sphinxcode{\sphinxupquote{\#}} to return the size of the sequence, i.e. the number of elements stored including the \sphinxcode{\sphinxupquote{"\textbackslash{}\$start"}} and
\sphinxcode{\sphinxupquote{"\textbackslash{}\$end"}} markers.

\sphinxlineitem{\sphinxstylestrong{\_\_index}}
\sphinxAtStartPar
A \sphinxstyleemphasis{method} \sphinxcode{\sphinxupquote{(key)}} called by the indexing operator \sphinxcode{\sphinxupquote{{[}key{]}}} to return the \sphinxstyleemphasis{value} of an attribute determined by \sphinxstyleemphasis{key}. The \sphinxstyleemphasis{key} is interpreted differently depending
on its type with the following precedence:
1. A \sphinxstyleemphasis{number} is interpreted as an element index and returns the element or \sphinxcode{\sphinxupquote{nil}}.
\#. Other \sphinxstyleemphasis{key} types are interpreted as \sphinxstyleemphasis{object} attributes subject to object model lookup.
\#. If the \sphinxstyleemphasis{value} associated with \sphinxstyleemphasis{key} is \sphinxcode{\sphinxupquote{nil}}, then \sphinxstyleemphasis{key} is interpreted as an element name and returns either the element or an \sphinxstyleemphasis{iterable} on the elements with the same name. %
\begin{footnote}[4]\sphinxAtStartFootnote
An \sphinxstyleemphasis{iterable} supports the length operator \sphinxcode{\sphinxupquote{\#}}, the indexing operator \sphinxcode{\sphinxupquote{{[}{]}}} and generic \sphinxcode{\sphinxupquote{for}} loops with \sphinxcode{\sphinxupquote{ipairs}}.
%
\end{footnote}
\#. Otherwise returns \sphinxcode{\sphinxupquote{nil}}.

\sphinxlineitem{\sphinxstylestrong{\_\_newindex}}
\sphinxAtStartPar
A \sphinxstyleemphasis{method} \sphinxcode{\sphinxupquote{(key, val)}} called by the assignment operator \sphinxcode{\sphinxupquote{{[}key{]}=val}} to create new attributes for the pairs (\sphinxstyleemphasis{key}, \sphinxstyleemphasis{value}).
If \sphinxstyleemphasis{key} is a \sphinxstyleemphasis{number} specifying the index or a \sphinxstyleemphasis{string} specifying the name of an existing element, the following error is raised:
\sphinxcode{\sphinxupquote{"invalid sequence write access (use replace method)"}}

\sphinxlineitem{\sphinxstylestrong{\_\_init}}
\sphinxAtStartPar
A \sphinxstyleemphasis{method} () called by the constructor to compute the elements positions.

\sphinxlineitem{\sphinxstylestrong{\_\_copy}}
\sphinxAtStartPar
A \sphinxstyleemphasis{method} () similar to the \sphinxcode{\sphinxupquote{:copy}} \sphinxstyleemphasis{method}.

\end{description}

\sphinxAtStartPar
The following attribute is stored with metamethods in the metatable, but has different purpose:

\sphinxAtStartPar
\sphinxstylestrong{\_\_sequ} A unique private \sphinxstyleemphasis{reference} that characterizes sequences.


\section{Sequences creation}
\label{\detokenize{sequences:sequences-creation}}
\sphinxAtStartPar
During its creation as an \sphinxstyleemphasis{object}, a sequence can defined its attributes as any object, and the \sphinxstyleemphasis{list} of its elements that must form a
\sphinxstyleemphasis{sequence} of increasing \(s\)\sphinxhyphen{}positions. When subsequences are part of this \sphinxstyleemphasis{list}, they are replaced by their respective elements as a
sequence \sphinxstyleemphasis{element} cannot be present inside other sequences. If the length of the sequence is not provided, it will be computed and set automatically.
During their creation, sequences compute the \(s\)\sphinxhyphen{}positions of their elements as described in the section {\hyperref[\detokenize{sequences:element-positions}]{\sphinxcrossref{element positions}}}, and check for overlapping
elements that would raise a “negative drift” runtime error.

\sphinxAtStartPar
The following example shows how to create a sequence form a \sphinxstyleemphasis{list} of elements and subsequences:

\begin{sphinxVerbatim}[commandchars=\\\{\}]
\PYG{k+kd}{local} \PYG{n}{sequence}\PYG{p}{,} \PYG{n}{drift}\PYG{p}{,} \PYG{n}{marker} \PYG{k+kr}{in} \PYG{n}{MAD}\PYG{p}{.}\PYG{n}{element}
\PYG{k+kd}{local} \PYG{n}{df}\PYG{p}{,} \PYG{n}{mk} \PYG{o}{=} \PYG{n}{drift} \PYG{l+s+s1}{\PYGZsq{}}\PYG{l+s+s1}{df}\PYG{l+s+s1}{\PYGZsq{}} \PYG{p}{\PYGZob{}}\PYG{n}{l}\PYG{o}{=}\PYG{l+m+mi}{1}\PYG{p}{\PYGZcb{}}\PYG{p}{,} \PYG{n}{marker} \PYG{l+s+s1}{\PYGZsq{}}\PYG{l+s+s1}{mk}\PYG{l+s+s1}{\PYGZsq{}} \PYG{p}{\PYGZob{}}\PYG{p}{\PYGZcb{}}
\PYG{k+kd}{local} \PYG{n}{seq} \PYG{o}{=} \PYG{n}{sequence} \PYG{l+s+s1}{\PYGZsq{}}\PYG{l+s+s1}{seq}\PYG{l+s+s1}{\PYGZsq{}} \PYG{p}{\PYGZob{}}
\PYG{n}{df} \PYG{l+s+s1}{\PYGZsq{}}\PYG{l+s+s1}{df1}\PYG{l+s+s1}{\PYGZsq{}} \PYG{p}{\PYGZob{}}\PYG{p}{\PYGZcb{}}\PYG{p}{,} \PYG{n}{mk} \PYG{l+s+s1}{\PYGZsq{}}\PYG{l+s+s1}{mk1}\PYG{l+s+s1}{\PYGZsq{}} \PYG{p}{\PYGZob{}}\PYG{p}{\PYGZcb{}}\PYG{p}{,}
\PYG{n}{sequence} \PYG{p}{\PYGZob{}}
   \PYG{n}{sequence} \PYG{p}{\PYGZob{}} \PYG{n}{mk} \PYG{l+s+s1}{\PYGZsq{}}\PYG{l+s+s1}{mk0}\PYG{l+s+s1}{\PYGZsq{}} \PYG{p}{\PYGZob{}}\PYG{p}{\PYGZcb{}} \PYG{p}{\PYGZcb{}}\PYG{p}{,}
   \PYG{n}{df} \PYG{l+s+s1}{\PYGZsq{}}\PYG{l+s+s1}{df.s}\PYG{l+s+s1}{\PYGZsq{}} \PYG{p}{\PYGZob{}}\PYG{p}{\PYGZcb{}}\PYG{p}{,} \PYG{n}{mk} \PYG{l+s+s1}{\PYGZsq{}}\PYG{l+s+s1}{mk.s}\PYG{l+s+s1}{\PYGZsq{}} \PYG{p}{\PYGZob{}}\PYG{p}{\PYGZcb{}}
\PYG{p}{\PYGZcb{}}\PYG{p}{,}
\PYG{n}{df} \PYG{l+s+s1}{\PYGZsq{}}\PYG{l+s+s1}{df2}\PYG{l+s+s1}{\PYGZsq{}} \PYG{p}{\PYGZob{}}\PYG{p}{\PYGZcb{}}\PYG{p}{,} \PYG{n}{mk} \PYG{l+s+s1}{\PYGZsq{}}\PYG{l+s+s1}{mk2}\PYG{l+s+s1}{\PYGZsq{}} \PYG{p}{\PYGZob{}}\PYG{p}{\PYGZcb{}}\PYG{p}{,}
\PYG{p}{\PYGZcb{}} \PYG{p}{:}\PYG{n}{dumpseq}\PYG{p}{(}\PYG{p}{)}
\end{sphinxVerbatim}

\sphinxAtStartPar
Displays

\begin{sphinxVerbatim}[commandchars=\\\{\}]
sequence: seq, l=3
idx  kind     name         l          dl       spos       upos    uds
001  marker   (*\PYGZdl{}start*)   0.000       0       0.000      0.000   0.000
002  drift    df1          1.000       0       0.000      0.500   0.500
003  marker   mk1          0.000       0       1.000      1.000   0.000
004  marker   mk0          0.000       0       1.000      1.000   0.000
005  drift    df.s         1.000       0       1.000      1.500   0.500
006  marker   mk.s         0.000       0       2.000      2.000   0.000
007  drift    df2          1.000       0       2.000      2.500   0.500
008  marker   mk2          0.000       0       3.000      3.000   0.000
009  marker   (*\PYGZdl{}end*)     0.000       0       3.000      3.000   0.000
\end{sphinxVerbatim}


\section{Element positions}
\label{\detokenize{sequences:element-positions}}
\sphinxAtStartPar
A sequence looks at the following attributes of an element, including sub\sphinxhyphen{}sequences, when installing it, \sphinxstyleemphasis{and only at that time}, to determine its position:
\begin{description}
\sphinxlineitem{\sphinxstylestrong{at}}
\sphinxAtStartPar
A \sphinxstyleemphasis{number} holding the position in {[}m{]} of the element in the sequence relative to the position specified by the \sphinxcode{\sphinxupquote{from}} attribute.

\sphinxlineitem{\sphinxstylestrong{from}}
\sphinxAtStartPar
A \sphinxstyleemphasis{string} holding one of \sphinxcode{\sphinxupquote{"start"}}, \sphinxcode{\sphinxupquote{"prev"}}, \sphinxcode{\sphinxupquote{"next"}}, \sphinxcode{\sphinxupquote{"end"}} or \sphinxcode{\sphinxupquote{"selected"}}, or the (mangled) name of another element to use as the reference position,
or a \sphinxstyleemphasis{number} holding a position in {[}m{]} from the start of the sequence. (default: \sphinxcode{\sphinxupquote{"start"}} if \sphinxcode{\sphinxupquote{at}}\(\geq 0\), \sphinxcode{\sphinxupquote{"end"}} if \sphinxcode{\sphinxupquote{at}}\(<0\), and \sphinxcode{\sphinxupquote{"prev"}}
otherwise)

\sphinxlineitem{\sphinxstylestrong{refpos}}
\sphinxAtStartPar
A \sphinxstyleemphasis{string} holding one of \sphinxcode{\sphinxupquote{"entry"}}, \sphinxcode{\sphinxupquote{"centre"}} or \sphinxcode{\sphinxupquote{"exit"}},  or the (mangled) name of a sequence sub\sphinxhyphen{}element to use as the reference position,
or a \sphinxstyleemphasis{number} specifying a position {[}m{]} from the start of the element, all of them resulting in an offset to substract to the \sphinxcode{\sphinxupquote{at}} attribute to find the
\(s\)\sphinxhyphen{}position of the element entry. (default: \sphinxcode{\sphinxupquote{nil}} \(\equiv\) \sphinxcode{\sphinxupquote{self.refer}}).

\sphinxlineitem{\sphinxstylestrong{shared}}
\sphinxAtStartPar
A \sphinxstyleemphasis{logical} specifying if an element is used at different positions in the same sequence definition, i.e. shared multiple times,
through temporary instances to store the many \sphinxcode{\sphinxupquote{at}} and \sphinxcode{\sphinxupquote{from}} attributes needed to specify its positions.
Once built, the sequence will drop these temporary instances in favor of their common parent, i.e. the original shared element.

\sphinxlineitem{\sphinxstylestrong{Warning:}}
\sphinxAtStartPar
The \sphinxcode{\sphinxupquote{at}} and \sphinxcode{\sphinxupquote{from}} attributes are not considered as intrinsic properties of the elements and are used only once during installation.
Any reuse of these attributes is the responsibility of the user, including the consistency between \sphinxcode{\sphinxupquote{at}} and \sphinxcode{\sphinxupquote{from}} after updates.

\end{description}


\section{Element selections}
\label{\detokenize{sequences:element-selections}}
\sphinxAtStartPar
The element selection in sequence use predicates in combination with iterators. The sequence iterator manages the range of elements where to apply the selection,
while the predicate says if an element in this range is illegible for the selection. In order to ease the use of methods based on the \sphinxcode{\sphinxupquote{:foreach}} method,
the selector predicate \sphinxcode{\sphinxupquote{sel}} can be built from different types of information provided in a \sphinxstyleemphasis{set} with the following attributes:
\begin{description}
\sphinxlineitem{\sphinxstylestrong{flag}}
\sphinxAtStartPar
A \sphinxstyleemphasis{number} interpreted as a flags mask to pass to the element method \sphinxcode{\sphinxupquote{:is\_selected}}. It should not be confused with the flags passed as argument to methods
\sphinxcode{\sphinxupquote{:select}} and \sphinxcode{\sphinxupquote{:deselect}}, as both flags can be used together but with different meanings!

\sphinxlineitem{\sphinxstylestrong{pattern}}
\sphinxAtStartPar
A \sphinxstyleemphasis{string} interpreted as a pattern to match the element name using \sphinxcode{\sphinxupquote{string.match}} from the standard library, see
\sphinxhref{http://github.com/MethodicalAcceleratorDesign/MADdocs/blob/master/lua52-refman-madng.pdf}{Lua 5.2} \S{}6.4 for details.

\sphinxlineitem{\sphinxstylestrong{class}}
\sphinxAtStartPar
An \sphinxstyleemphasis{element} interpreted as a \sphinxstyleemphasis{class} to pass to the element method \sphinxcode{\sphinxupquote{:is\_instansceOf}}.

\sphinxlineitem{\sphinxstylestrong{list}}
\sphinxAtStartPar
An \sphinxstyleemphasis{iterable} interpreted as a \sphinxstyleemphasis{list} used to build a \sphinxstyleemphasis{set} and select the elements by their name, i.e. the built predicate will use \sphinxcode{\sphinxupquote{tbl{[}elm.name{]}}}
as a \sphinxstyleemphasis{logical}. If the \sphinxstyleemphasis{iterable} is a single item, e.g. a \sphinxstyleemphasis{string}, it will be converted first to a \sphinxstyleemphasis{list}.

\sphinxlineitem{\sphinxstylestrong{table}}
\sphinxAtStartPar
A \sphinxstyleemphasis{mappable} interpreted as a \sphinxstyleemphasis{set} used to select the elements by their name, i.e. the built predicate will use \sphinxcode{\sphinxupquote{tbl{[}elm.name{]}}} as a \sphinxstyleemphasis{logical}.
If the \sphinxstyleemphasis{mappable} contains a \sphinxstyleemphasis{list} or is a single item, it will be converted first to a \sphinxstyleemphasis{list} and its \sphinxstyleemphasis{set} part will be discarded.

\sphinxlineitem{\sphinxstylestrong{select}}
\sphinxAtStartPar
A \sphinxstyleemphasis{callable} interpreted as the selector itself, which allows to build any kind of predicate or to complete the restrictions already built above.

\sphinxlineitem{\sphinxstylestrong{subelem}}
\sphinxAtStartPar
A \sphinxstyleemphasis{boolean} indicating to include or not the sub\sphinxhyphen{}elements in the scanning loop. The predicate and the action receive the sub\sphinxhyphen{}element and its sub\sphinxhyphen{}index as
first and second argument, and the main element index as third argument.

\end{description}

\sphinxAtStartPar
All these attributes are used in the aforementioned order to incrementally build predicates that are combined with logical conjunctions, i.e. \sphinxcode{\sphinxupquote{and}}’ed,
to give the final predicate used by the \sphinxcode{\sphinxupquote{:foreach}} method. If only one of these attributes is needed, it is possible to pass it directly in \sphinxcode{\sphinxupquote{sel}},
not as an attribute in a \sphinxstyleemphasis{set}, and its type will be used to determine the kind of predicate to build. For example, \sphinxcode{\sphinxupquote{self:foreach(act, monitor)}} is equivalent
to \sphinxcode{\sphinxupquote{self:foreach\textbackslash{}\{action=act, class=monitor\}}}.


\section{Indexes, names and counts}
\label{\detokenize{sequences:indexes-names-and-counts}}
\sphinxAtStartPar
Indexing a sequence triggers a complex look up mechanism where the arguments will be interpreted in various ways as described in the \sphinxcode{\sphinxupquote{:\_\_index}} metamethod.
A \sphinxstyleemphasis{number} will be interpreted as a relative slot index in the list of elements, and a negative index will be considered as relative to the end of the sequence,
i.e. \sphinxcode{\sphinxupquote{\sphinxhyphen{}1}} is the \sphinxcode{\sphinxupquote{\$end}} marker. Non\sphinxhyphen{} \sphinxstyleemphasis{number} will be interpreted first as an object key (can be anything), looking for sequence methods or attributes;
then as an element name if nothing was found.

\sphinxAtStartPar
If an element exists but its name is not unique in the sequence, an \sphinxstyleemphasis{iterable} is returned. An \sphinxstyleemphasis{iterable} supports the length \sphinxcode{\sphinxupquote{\#}} operator to retrieve the
number of elements with the same name, the indexing operator \sphinxcode{\sphinxupquote{{[}{]}}} waiting for a count \$n\$ to retrieve the \(n\)\sphinxhyphen{}th element from the start with that name,
and the iterator \sphinxcode{\sphinxupquote{ipairs}} to use with generic \sphinxcode{\sphinxupquote{for}} loops.

\sphinxAtStartPar
The returned \sphinxstyleemphasis{iterable} is in practice a proxy, i.e. a fake intermediate object that emulates the expected behavior, and any attempt to access the proxy in
another manner should raise a runtime error.

\sphinxAtStartPar
\sphinxstylestrong{Warning:} The indexing operator \sphinxcode{\sphinxupquote{{[}{]}}} interprets a \sphinxstyleemphasis{number} as a (relative) element index as the method \sphinxcode{\sphinxupquote{:index}}, while the method \sphinxcode{\sphinxupquote{:index\_of}}\} interprets a
\sphinxstyleemphasis{number} as a (relative) element \(s\)\sphinxhyphen{}position {[}m{]}.

\sphinxAtStartPar
The following example shows how to access to the elements through indexing and the \sphinxstyleemphasis{iterable}::

\begin{sphinxVerbatim}[commandchars=\\\{\}]
\PYG{k+kd}{local} \PYG{n}{sequence}\PYG{p}{,} \PYG{n}{drift}\PYG{p}{,} \PYG{n}{marker} \PYG{k+kr}{in} \PYG{n}{MAD}\PYG{p}{.}\PYG{n}{element}
\PYG{k+kd}{local} \PYG{n}{seq} \PYG{o}{=} \PYG{n}{sequence} \PYG{p}{\PYGZob{}}
\PYG{n}{drift} \PYG{l+s+s1}{\PYGZsq{}}\PYG{l+s+s1}{df}\PYG{l+s+s1}{\PYGZsq{}} \PYG{p}{\PYGZob{}} \PYG{n}{id}\PYG{o}{=}\PYG{l+m+mi}{1} \PYG{p}{\PYGZcb{}}\PYG{p}{,} \PYG{n}{marker} \PYG{l+s+s1}{\PYGZsq{}}\PYG{l+s+s1}{mk}\PYG{l+s+s1}{\PYGZsq{}} \PYG{p}{\PYGZob{}} \PYG{n}{id}\PYG{o}{=}\PYG{l+m+mi}{2} \PYG{p}{\PYGZcb{}}\PYG{p}{,}
\PYG{n}{drift} \PYG{l+s+s1}{\PYGZsq{}}\PYG{l+s+s1}{df}\PYG{l+s+s1}{\PYGZsq{}} \PYG{p}{\PYGZob{}} \PYG{n}{id}\PYG{o}{=}\PYG{l+m+mi}{3} \PYG{p}{\PYGZcb{}}\PYG{p}{,} \PYG{n}{marker} \PYG{l+s+s1}{\PYGZsq{}}\PYG{l+s+s1}{mk}\PYG{l+s+s1}{\PYGZsq{}} \PYG{p}{\PYGZob{}} \PYG{n}{id}\PYG{o}{=}\PYG{l+m+mi}{4} \PYG{p}{\PYGZcb{}}\PYG{p}{,}
\PYG{n}{drift} \PYG{l+s+s1}{\PYGZsq{}}\PYG{l+s+s1}{df}\PYG{l+s+s1}{\PYGZsq{}} \PYG{p}{\PYGZob{}} \PYG{n}{id}\PYG{o}{=}\PYG{l+m+mi}{5} \PYG{p}{\PYGZcb{}}\PYG{p}{,} \PYG{n}{marker} \PYG{l+s+s1}{\PYGZsq{}}\PYG{l+s+s1}{mk}\PYG{l+s+s1}{\PYGZsq{}} \PYG{p}{\PYGZob{}} \PYG{n}{id}\PYG{o}{=}\PYG{l+m+mi}{6} \PYG{p}{\PYGZcb{}}\PYG{p}{,}
\PYG{p}{\PYGZcb{}}
\PYG{n+nb}{print}\PYG{p}{(}\PYG{n}{seq}\PYG{p}{[} \PYG{l+m+mi}{1}\PYG{p}{]}\PYG{p}{.}\PYG{n}{name}\PYG{p}{)} \PYG{c+c1}{\PYGZhy{}\PYGZhy{} display: (*\PYGZbs{}\PYGZdl{}start*) (start marker)}
\PYG{n+nb}{print}\PYG{p}{(}\PYG{n}{seq}\PYG{p}{[}\PYG{o}{\PYGZhy{}}\PYG{l+m+mi}{1}\PYG{p}{]}\PYG{p}{.}\PYG{n}{name}\PYG{p}{)} \PYG{c+c1}{\PYGZhy{}\PYGZhy{} display: (*\PYGZbs{}\PYGZdl{}end*)   (end   marker)}

\PYG{n+nb}{print}\PYG{p}{(}\PYG{o}{\PYGZsh{}}\PYG{n}{seq}\PYG{p}{.}\PYG{n}{df}\PYG{p}{,} \PYG{n}{seq}\PYG{p}{.}\PYG{n}{df}\PYG{p}{[}\PYG{l+m+mi}{3}\PYG{p}{]}\PYG{p}{.}\PYG{n}{id}\PYG{p}{)}                        \PYG{c+c1}{\PYGZhy{}\PYGZhy{} display: 3   5}
\PYG{k+kr}{for} \PYG{n}{\PYGZus{}}\PYG{p}{,}\PYG{n}{e} \PYG{k+kr}{in} \PYG{n+nb}{ipairs}\PYG{p}{(}\PYG{n}{seq}\PYG{p}{.}\PYG{n}{df}\PYG{p}{)} \PYG{k+kr}{do} \PYG{n+nb}{io.write}\PYG{p}{(}\PYG{n}{e}\PYG{p}{.}\PYG{n}{id}\PYG{p}{,}\PYG{l+s+s2}{\PYGZdq{}}\PYG{l+s+s2}{ }\PYG{l+s+s2}{\PYGZdq{}}\PYG{p}{)} \PYG{k+kr}{end} \PYG{c+c1}{\PYGZhy{}\PYGZhy{} display: 1 3 5}
\PYG{k+kr}{for} \PYG{n}{\PYGZus{}}\PYG{p}{,}\PYG{n}{e} \PYG{k+kr}{in} \PYG{n+nb}{ipairs}\PYG{p}{(}\PYG{n}{seq}\PYG{p}{.}\PYG{n}{mk}\PYG{p}{)} \PYG{k+kr}{do} \PYG{n+nb}{io.write}\PYG{p}{(}\PYG{n}{e}\PYG{p}{.}\PYG{n}{id}\PYG{p}{,}\PYG{l+s+s2}{\PYGZdq{}}\PYG{l+s+s2}{ }\PYG{l+s+s2}{\PYGZdq{}}\PYG{p}{)} \PYG{k+kr}{end} \PYG{c+c1}{\PYGZhy{}\PYGZhy{} display: 2 4 6}

\PYG{c+c1}{\PYGZhy{}\PYGZhy{} print name of drift with id=3 in absolute and relative to id=6.}
\PYG{n+nb}{print}\PYG{p}{(}\PYG{n}{seq}\PYG{p}{:}\PYG{n}{name\PYGZus{}of}\PYG{p}{(}\PYG{l+m+mi}{4}\PYG{p}{)}\PYG{p}{)}       \PYG{c+c1}{\PYGZhy{}\PYGZhy{} display: df[2]  (2nd df from start)}
\PYG{n+nb}{print}\PYG{p}{(}\PYG{n}{seq}\PYG{p}{:}\PYG{n}{name\PYGZus{}of}\PYG{p}{(}\PYG{l+m+mi}{2}\PYG{p}{,} \PYG{o}{\PYGZhy{}}\PYG{l+m+mi}{2}\PYG{p}{)}\PYG{p}{)}   \PYG{c+c1}{\PYGZhy{}\PYGZhy{} display: df\PYGZob{}\PYGZhy{}3\PYGZcb{} (3rd df before last mk)}
\end{sphinxVerbatim}

\sphinxAtStartPar
The last two lines of code display the name of the same element but mangled with absolute and relative counts.

\sphinxAtStartPar
section\{Iterators and ranges\}

\sphinxAtStartPar
Ranging a sequence triggers a complex look up mechanism where the arguments will be interpreted in various ways as described in the \sphinxcode{\sphinxupquote{:range\_of}} method,
itself based on the methods \sphinxcode{\sphinxupquote{:index\_of}}\} and \sphinxcode{\sphinxupquote{:index}}. The number of elements selected by a sequence range can be computed by the \sphinxcode{\sphinxupquote{:length\_of}}\} method,
which accepts an extra \sphinxstyleemphasis{number} of turns to consider in the calculation.

\sphinxAtStartPar
The sequence iterators are created by the methods \sphinxcode{\sphinxupquote{:iter}} and \sphinxcode{\sphinxupquote{:siter}}, and both are based on the \sphinxcode{\sphinxupquote{:range\_of}} method as mentioned in their descriptions
and includes an extra \sphinxstyleemphasis{number} of turns as for the method \sphinxcode{\sphinxupquote{:length\_of}}, and a direction \sphinxcode{\sphinxupquote{1}} (forward) or \sphinxcode{\sphinxupquote{\sphinxhyphen{}1}} (backward) for the iteration.
The \sphinxcode{\sphinxupquote{:siter}} differs from the \sphinxcode{\sphinxupquote{:iter}} by its loop, which returns not only the sequence elements but also \sphinxstyleemphasis{implicit} drifts built on\sphinxhyphen{}the\sphinxhyphen{}fly when a gap
\(>10^{-10}\) m is detected between two sequence elements. Such implicit drift have half\sphinxhyphen{}integer indexes and make the iterator “continuous” in \(s\)\sphinxhyphen{}positions.

\sphinxAtStartPar
The method \sphinxcode{\sphinxupquote{:foreach}} uses the iterator returned by \sphinxcode{\sphinxupquote{:iter}} with a range as its sole argument to loop over the elements where to apply the predicate before
executing the action. The methods \sphinxcode{\sphinxupquote{:select}}, \sphinxcode{\sphinxupquote{:deselect}}, \sphinxcode{\sphinxupquote{:filter}}, \sphinxcode{\sphinxupquote{:install}}, \sphinxcode{\sphinxupquote{:replace}}, \sphinxcode{\sphinxupquote{:remove}}, \sphinxcode{\sphinxupquote{:move}}, and \sphinxcode{\sphinxupquote{:misalign}} are all based
directly or indirectly on the \sphinxcode{\sphinxupquote{:foreach}} method. Hence, to iterate backward over a sequence range, these methods have to use either its \sphinxstyleemphasis{list} form or a numerical range.
For example the invocation \sphinxcode{\sphinxupquote{seq:foreach(\textbackslash{}e \sphinxhyphen{}\textgreater{} print(e.name), \{2, 2, \textquotesingle{}idx\textquotesingle{}, \sphinxhyphen{}1)}} will iterate backward over the entire sequence \sphinxcode{\sphinxupquote{seq}} excluding the \sphinxcode{\sphinxupquote{\$start}}
and \sphinxcode{\sphinxupquote{\$end}} markers, while the invocation \sphinxcode{\sphinxupquote{seq:foreach(\textbackslash{}e \sphinxhyphen{}\textgreater{} print(e.name), 5..2..\sphinxhyphen{}1)}} will iterate backward over the elements with \(s\)\sphinxhyphen{}positions sitting in the
interval \([2,5]\) m.

\sphinxAtStartPar
The tracking commands \sphinxcode{\sphinxupquote{survey}} and \sphinxcode{\sphinxupquote{track}} use the iterator returned by \sphinxcode{\sphinxupquote{:siter}} for their main loop, with their \sphinxcode{\sphinxupquote{range}}, \sphinxcode{\sphinxupquote{nturn}} and \sphinxcode{\sphinxupquote{dir}} attributes
as arguments. These commands also save the iterator states in their \sphinxcode{\sphinxupquote{mflw}} to allow the users to run them \sphinxcode{\sphinxupquote{nstep}} by \sphinxcode{\sphinxupquote{nstep}}, see commands {\hyperref[\detokenize{survey::doc}]{\sphinxcrossref{\DUrole{doc}{survey}}}}
and {\hyperref[\detokenize{track::doc}]{\sphinxcrossref{\DUrole{doc}{track}}}} for details.

\sphinxAtStartPar
The following example shows how to access to the elements with the \sphinxcode{\sphinxupquote{:foreach}} method::

\begin{sphinxVerbatim}[commandchars=\\\{\}]
local sequence, drift, marker in MAD.element
local observed in MAD.element.flags
local seq = sequence \PYGZob{}
drift \PYGZsq{}df\PYGZsq{} \PYGZob{} id=1 \PYGZcb{}, marker \PYGZsq{}mk\PYGZsq{} \PYGZob{} id=2 \PYGZcb{},
drift \PYGZsq{}df\PYGZsq{} \PYGZob{} id=3 \PYGZcb{}, marker \PYGZsq{}mk\PYGZsq{} \PYGZob{} id=4 \PYGZcb{},
drift \PYGZsq{}df\PYGZsq{} \PYGZob{} id=5 \PYGZcb{}, marker \PYGZsq{}mk\PYGZsq{} \PYGZob{} id=6 \PYGZcb{},
\PYGZcb{}

local act = \PYGZbs{}e \PYGZhy{}\PYGZgt{} print(e.name,e.id)
seq:foreach(act, \PYGZdq{}df[2]/mk[3]\PYGZdq{})
\PYGZhy{}\PYGZhy{} display:
df   3
mk   4
df   5
mk   6

seq:foreach\PYGZob{}action=act, range=\PYGZdq{}df[2]/mk[3]\PYGZdq{}, class=marker\PYGZcb{}
\PYGZhy{}\PYGZhy{} display: markers at ids 4 and 6
seq:foreach\PYGZob{}action=act, pattern=(*\PYGZbs{}verb+\PYGZdq{}\PYGZca{}[\PYGZca{}\PYGZdl{}]\PYGZdq{}+*)\PYGZcb{}
\PYGZhy{}\PYGZhy{} display: all elements except (*\PYGZbs{}verb+\PYGZdl{}start and \PYGZdl{}end+*) markers
seq:foreach\PYGZob{}action=\PYGZbs{}e \PYGZhy{}\PYGZgt{} e:select(observed), pattern=\PYGZdq{}mk\PYGZdq{}\PYGZcb{}
\PYGZhy{}\PYGZhy{} same as: seq:select(observed, \PYGZob{}pattern=\PYGZdq{}mk\PYGZdq{}\PYGZcb{})

local act = \PYGZbs{}e \PYGZhy{}\PYGZgt{} print(e.name, e.id, e:is\PYGZus{}observed())
seq:foreach\PYGZob{}action=act, range=(*\PYGZbs{}verb+\PYGZdq{}\PYGZsh{}s/\PYGZsh{}e\PYGZdq{}+*)\PYGZcb{}
\PYGZhy{}\PYGZhy{} display:
(*\PYGZbs{}\PYGZdl{}start*)   nil  false
df       1    false
mk       2    true
df       3    false
mk       4    true
df       5    false
mk       6    true
(*\PYGZbs{}\PYGZdl{}end*)     nil  true
\end{sphinxVerbatim}


\section{Examples}
\label{\detokenize{sequences:examples}}

\subsection{FODO cell}
\label{\detokenize{sequences:fodo-cell}}
\begin{sphinxVerbatim}[commandchars=\\\{\}]
\PYG{k+kd}{local} \PYG{n}{sequence}\PYG{p}{,} \PYG{n}{sbend}\PYG{p}{,} \PYG{n}{quadrupole}\PYG{p}{,} \PYG{n}{sextupole}\PYG{p}{,} \PYG{n}{hkicker}\PYG{p}{,} \PYG{n}{vkicker}\PYG{p}{,} \PYG{n}{marker} \PYG{k+kr}{in} \PYG{n}{MAD}\PYG{p}{.}\PYG{n}{element}
\PYG{k+kd}{local} \PYG{n}{mkf} \PYG{o}{=} \PYG{n}{marker} \PYG{l+s+s1}{\PYGZsq{}}\PYG{l+s+s1}{mkf}\PYG{l+s+s1}{\PYGZsq{}} \PYG{p}{\PYGZob{}}\PYG{p}{\PYGZcb{}}
\PYG{k+kd}{local} \PYG{n}{ang}\PYG{o}{=}\PYG{l+m+mi}{2}\PYG{o}{*}\PYG{n+nb}{math.pi}\PYG{o}{/}\PYG{l+m+mi}{80}
\PYG{k+kd}{local} \PYG{n}{fodo} \PYG{o}{=} \PYG{n}{sequence} \PYG{l+s+s1}{\PYGZsq{}}\PYG{l+s+s1}{fodo}\PYG{l+s+s1}{\PYGZsq{}} \PYG{p}{\PYGZob{}} \PYG{n}{refer}\PYG{o}{=}\PYG{l+s+s1}{\PYGZsq{}}\PYG{l+s+s1}{entry}\PYG{l+s+s1}{\PYGZsq{}}\PYG{p}{,}
\PYG{n}{mkf}             \PYG{p}{\PYGZob{}} \PYG{n}{at}\PYG{o}{=}\PYG{l+m+mi}{0}\PYG{p}{,} \PYG{n}{shared}\PYG{o}{=}\PYG{k+kc}{true}      \PYG{p}{\PYGZcb{}}\PYG{p}{,} \PYG{c+c1}{\PYGZhy{}\PYGZhy{} mark the start of the fodo}
\PYG{n}{quadrupole} \PYG{l+s+s1}{\PYGZsq{}}\PYG{l+s+s1}{qf}\PYG{l+s+s1}{\PYGZsq{}} \PYG{p}{\PYGZob{}} \PYG{n}{at}\PYG{o}{=}\PYG{l+m+mi}{0}\PYG{p}{,} \PYG{n}{l}\PYG{o}{=}\PYG{l+m+mi}{1}  \PYG{p}{,} \PYG{n}{k1}\PYG{o}{=}\PYG{l+m+mf}{0.3}    \PYG{p}{\PYGZcb{}}\PYG{p}{,}
\PYG{n}{sextupole}  \PYG{l+s+s1}{\PYGZsq{}}\PYG{l+s+s1}{sf}\PYG{l+s+s1}{\PYGZsq{}} \PYG{p}{\PYGZob{}}       \PYG{n}{l}\PYG{o}{=}\PYG{l+m+mf}{0.3}\PYG{p}{,} \PYG{n}{k2}\PYG{o}{=}\PYG{l+m+mi}{0}      \PYG{p}{\PYGZcb{}}\PYG{p}{,}
\PYG{n}{hkicker}    \PYG{l+s+s1}{\PYGZsq{}}\PYG{l+s+s1}{hk}\PYG{l+s+s1}{\PYGZsq{}} \PYG{p}{\PYGZob{}}       \PYG{n}{l}\PYG{o}{=}\PYG{l+m+mf}{0.2}\PYG{p}{,} \PYG{n}{kick}\PYG{o}{=}\PYG{l+m+mi}{0}    \PYG{p}{\PYGZcb{}}\PYG{p}{,}
\PYG{n}{sbend}      \PYG{l+s+s1}{\PYGZsq{}}\PYG{l+s+s1}{mb}\PYG{l+s+s1}{\PYGZsq{}} \PYG{p}{\PYGZob{}} \PYG{n}{at}\PYG{o}{=}\PYG{l+m+mi}{2}\PYG{p}{,} \PYG{n}{l}\PYG{o}{=}\PYG{l+m+mi}{2}  \PYG{p}{,} \PYG{n}{angle}\PYG{o}{=}\PYG{n}{ang} \PYG{p}{\PYGZcb{}}\PYG{p}{,}

\PYG{n}{quadrupole} \PYG{l+s+s1}{\PYGZsq{}}\PYG{l+s+s1}{qd}\PYG{l+s+s1}{\PYGZsq{}} \PYG{p}{\PYGZob{}} \PYG{n}{at}\PYG{o}{=}\PYG{l+m+mi}{5}\PYG{p}{,} \PYG{n}{l}\PYG{o}{=}\PYG{l+m+mi}{1}  \PYG{p}{,} \PYG{n}{k1}\PYG{o}{=\PYGZhy{}}\PYG{l+m+mf}{0.3}   \PYG{p}{\PYGZcb{}}\PYG{p}{,}
\PYG{n}{sextupole}  \PYG{l+s+s1}{\PYGZsq{}}\PYG{l+s+s1}{sd}\PYG{l+s+s1}{\PYGZsq{}} \PYG{p}{\PYGZob{}}       \PYG{n}{l}\PYG{o}{=}\PYG{l+m+mf}{0.3}\PYG{p}{,} \PYG{n}{k2}\PYG{o}{=}\PYG{l+m+mi}{0}      \PYG{p}{\PYGZcb{}}\PYG{p}{,}
\PYG{n}{vkicker}    \PYG{l+s+s1}{\PYGZsq{}}\PYG{l+s+s1}{vk}\PYG{l+s+s1}{\PYGZsq{}} \PYG{p}{\PYGZob{}}       \PYG{n}{l}\PYG{o}{=}\PYG{l+m+mf}{0.2}\PYG{p}{,} \PYG{n}{kick}\PYG{o}{=}\PYG{l+m+mi}{0}    \PYG{p}{\PYGZcb{}}\PYG{p}{,}
\PYG{n}{sbend}      \PYG{l+s+s1}{\PYGZsq{}}\PYG{l+s+s1}{mb}\PYG{l+s+s1}{\PYGZsq{}} \PYG{p}{\PYGZob{}} \PYG{n}{at}\PYG{o}{=}\PYG{l+m+mi}{7}\PYG{p}{,} \PYG{n}{l}\PYG{o}{=}\PYG{l+m+mi}{2}  \PYG{p}{,} \PYG{n}{angle}\PYG{o}{=}\PYG{n}{ang} \PYG{p}{\PYGZcb{}}\PYG{p}{,}
\PYG{p}{\PYGZcb{}}
\PYG{k+kd}{local} \PYG{n}{arc} \PYG{o}{=} \PYG{n}{sequence} \PYG{l+s+s1}{\PYGZsq{}}\PYG{l+s+s1}{arc}\PYG{l+s+s1}{\PYGZsq{}} \PYG{p}{\PYGZob{}} \PYG{n}{refer}\PYG{o}{=}\PYG{l+s+s1}{\PYGZsq{}}\PYG{l+s+s1}{entry}\PYG{l+s+s1}{\PYGZsq{}}\PYG{p}{,} \PYG{l+m+mi}{10}\PYG{o}{*}\PYG{n}{fodo} \PYG{p}{\PYGZcb{}}
\PYG{n}{fodo}\PYG{p}{:}\PYG{n}{dumpseq}\PYG{p}{(}\PYG{p}{)} \PYG{p}{;} \PYG{n+nb}{print}\PYG{p}{(}\PYG{n}{fodo}\PYG{p}{.}\PYG{n}{mkf}\PYG{p}{,} \PYG{n}{mkf}\PYG{p}{)}
\end{sphinxVerbatim}

\sphinxAtStartPar
Display:

\begin{sphinxVerbatim}[commandchars=\\\{\}]
sequence: fodo, l=9
idx  kind          name          l          dl       spos       upos    uds
001  marker        \PYGZdl{}start  0.000       0       0.000      0.000   0.000
002  marker        mkf     0.000       0       0.000      0.000   0.000
003  quadrupole    qf      1.000       0       0.000      0.000   0.000
004  sextupole     sf      0.300       0       1.000      1.000   0.000
005  hkicker       hk      0.200       0       1.300      1.300   0.000
006  sbend         mb      2.000       0       2.000      2.000   0.000
007  quadrupole    qd      1.000       0       5.000      5.000   0.000
008  sextupole     sd      0.300       0       6.000      6.000   0.000
009  vkicker       vk      0.200       0       6.300      6.300   0.000
010  sbend         mb      2.000       0       7.000      7.000   0.000
011  marker        \PYGZdl{}end    0.000       0       9.000      9.000   0.000
marker : \PYGZsq{}mkf\PYGZsq{} 0x01015310e8  marker: \PYGZsq{}mkf\PYGZsq{} 0x01015310e8 \PYGZhy{}\PYGZhy{} same marker
\end{sphinxVerbatim}


\subsection{SPS compact description}
\label{\detokenize{sequences:sps-compact-description}}
\sphinxAtStartPar
The following dummy example shows a compact definition of the SPS mixing elements, beam lines and sequence definitions.
The elements are zero\sphinxhyphen{}length, so the lattice is too.

\begin{sphinxVerbatim}[commandchars=\\\{\}]
\PYG{k+kd}{local} \PYG{n}{drift}\PYG{p}{,} \PYG{n}{sbend}\PYG{p}{,} \PYG{n}{quadrupole}\PYG{p}{,} \PYG{n}{bline}\PYG{p}{,} \PYG{n}{sequence} \PYG{k+kr}{in} \PYG{n}{MAD}\PYG{p}{.}\PYG{n}{element}

\PYG{c+c1}{\PYGZhy{}\PYGZhy{} elements (empty!)}
\PYG{k+kd}{local} \PYG{n}{ds} \PYG{o}{=} \PYG{n}{drift}      \PYG{l+s+s1}{\PYGZsq{}}\PYG{l+s+s1}{ds}\PYG{l+s+s1}{\PYGZsq{}} \PYG{p}{\PYGZob{}}\PYG{p}{\PYGZcb{}}
\PYG{k+kd}{local} \PYG{n}{dl} \PYG{o}{=} \PYG{n}{drift}      \PYG{l+s+s1}{\PYGZsq{}}\PYG{l+s+s1}{dl}\PYG{l+s+s1}{\PYGZsq{}} \PYG{p}{\PYGZob{}}\PYG{p}{\PYGZcb{}}
\PYG{k+kd}{local} \PYG{n}{dm} \PYG{o}{=} \PYG{n}{drift}      \PYG{l+s+s1}{\PYGZsq{}}\PYG{l+s+s1}{dm}\PYG{l+s+s1}{\PYGZsq{}} \PYG{p}{\PYGZob{}}\PYG{p}{\PYGZcb{}}
\PYG{k+kd}{local} \PYG{n}{b1} \PYG{o}{=} \PYG{n}{sbend}      \PYG{l+s+s1}{\PYGZsq{}}\PYG{l+s+s1}{b1}\PYG{l+s+s1}{\PYGZsq{}} \PYG{p}{\PYGZob{}}\PYG{p}{\PYGZcb{}}
\PYG{k+kd}{local} \PYG{n}{b2} \PYG{o}{=} \PYG{n}{sbend}      \PYG{l+s+s1}{\PYGZsq{}}\PYG{l+s+s1}{b2}\PYG{l+s+s1}{\PYGZsq{}} \PYG{p}{\PYGZob{}}\PYG{p}{\PYGZcb{}}
\PYG{k+kd}{local} \PYG{n}{qf} \PYG{o}{=} \PYG{n}{quadrupole} \PYG{l+s+s1}{\PYGZsq{}}\PYG{l+s+s1}{qf}\PYG{l+s+s1}{\PYGZsq{}} \PYG{p}{\PYGZob{}}\PYG{p}{\PYGZcb{}}
\PYG{k+kd}{local} \PYG{n}{qd} \PYG{o}{=} \PYG{n}{quadrupole} \PYG{l+s+s1}{\PYGZsq{}}\PYG{l+s+s1}{qd}\PYG{l+s+s1}{\PYGZsq{}} \PYG{p}{\PYGZob{}}\PYG{p}{\PYGZcb{}}

\PYG{c+c1}{\PYGZhy{}\PYGZhy{} subsequences}
\PYG{k+kd}{local} \PYG{n}{pf}  \PYG{o}{=} \PYG{n}{bline} \PYG{l+s+s1}{\PYGZsq{}}\PYG{l+s+s1}{pf}\PYG{l+s+s1}{\PYGZsq{}}  \PYG{p}{\PYGZob{}}\PYG{n}{qf}\PYG{p}{,}\PYG{l+m+mi}{2}\PYG{o}{*}\PYG{n}{b1}\PYG{p}{,}\PYG{l+m+mi}{2}\PYG{o}{*}\PYG{n}{b2}\PYG{p}{,}\PYG{n}{ds}\PYG{p}{\PYGZcb{}}           \PYG{c+c1}{\PYGZhy{}\PYGZhy{} \PYGZsh{}: 6}
\PYG{k+kd}{local} \PYG{n}{pd}  \PYG{o}{=} \PYG{n}{bline} \PYG{l+s+s1}{\PYGZsq{}}\PYG{l+s+s1}{pd}\PYG{l+s+s1}{\PYGZsq{}}  \PYG{p}{\PYGZob{}}\PYG{n}{qd}\PYG{p}{,}\PYG{l+m+mi}{2}\PYG{o}{*}\PYG{n}{b2}\PYG{p}{,}\PYG{l+m+mi}{2}\PYG{o}{*}\PYG{n}{b1}\PYG{p}{,}\PYG{n}{ds}\PYG{p}{\PYGZcb{}}           \PYG{c+c1}{\PYGZhy{}\PYGZhy{} \PYGZsh{}: 6}
\PYG{k+kd}{local} \PYG{n}{p24} \PYG{o}{=} \PYG{n}{bline} \PYG{l+s+s1}{\PYGZsq{}}\PYG{l+s+s1}{p24}\PYG{l+s+s1}{\PYGZsq{}} \PYG{p}{\PYGZob{}}\PYG{n}{qf}\PYG{p}{,}\PYG{n}{dm}\PYG{p}{,}\PYG{l+m+mi}{2}\PYG{o}{*}\PYG{n}{b2}\PYG{p}{,}\PYG{n}{ds}\PYG{p}{,}\PYG{n}{pd}\PYG{p}{\PYGZcb{}}          \PYG{c+c1}{\PYGZhy{}\PYGZhy{} \PYGZsh{}: 11 (5+6)}
\PYG{k+kd}{local} \PYG{n}{p42} \PYG{o}{=} \PYG{n}{bline} \PYG{l+s+s1}{\PYGZsq{}}\PYG{l+s+s1}{p42}\PYG{l+s+s1}{\PYGZsq{}} \PYG{p}{\PYGZob{}}\PYG{n}{pf}\PYG{p}{,}\PYG{n}{qd}\PYG{p}{,}\PYG{l+m+mi}{2}\PYG{o}{*}\PYG{n}{b2}\PYG{p}{,}\PYG{n}{dm}\PYG{p}{,}\PYG{n}{ds}\PYG{p}{\PYGZcb{}}          \PYG{c+c1}{\PYGZhy{}\PYGZhy{} \PYGZsh{}: 11 (6+5)}
\PYG{k+kd}{local} \PYG{n}{p00} \PYG{o}{=} \PYG{n}{bline} \PYG{l+s+s1}{\PYGZsq{}}\PYG{l+s+s1}{p00}\PYG{l+s+s1}{\PYGZsq{}} \PYG{p}{\PYGZob{}}\PYG{n}{qf}\PYG{p}{,}\PYG{n}{dl}\PYG{p}{,}\PYG{n}{qd}\PYG{p}{,}\PYG{n}{dl}\PYG{p}{\PYGZcb{}}               \PYG{c+c1}{\PYGZhy{}\PYGZhy{} \PYGZsh{}: 4}
\PYG{k+kd}{local} \PYG{n}{p44} \PYG{o}{=} \PYG{n}{bline} \PYG{l+s+s1}{\PYGZsq{}}\PYG{l+s+s1}{p44}\PYG{l+s+s1}{\PYGZsq{}} \PYG{p}{\PYGZob{}}\PYG{n}{pf}\PYG{p}{,}\PYG{n}{pd}\PYG{p}{\PYGZcb{}}                     \PYG{c+c1}{\PYGZhy{}\PYGZhy{} \PYGZsh{}: 12 (6+6)}
\PYG{k+kd}{local} \PYG{n}{insert} \PYG{o}{=} \PYG{n}{bline} \PYG{l+s+s1}{\PYGZsq{}}\PYG{l+s+s1}{insert}\PYG{l+s+s1}{\PYGZsq{}} \PYG{p}{\PYGZob{}}\PYG{n}{p24}\PYG{p}{,}\PYG{l+m+mi}{2}\PYG{o}{*}\PYG{n}{p00}\PYG{p}{,}\PYG{n}{p42}\PYG{p}{\PYGZcb{}}       \PYG{c+c1}{\PYGZhy{}\PYGZhy{} \PYGZsh{}: 30 (11+2*4+11)}
\PYG{k+kd}{local} \PYG{n}{super}  \PYG{o}{=} \PYG{n}{bline} \PYG{l+s+s1}{\PYGZsq{}}\PYG{l+s+s1}{super}\PYG{l+s+s1}{\PYGZsq{}}  \PYG{p}{\PYGZob{}}\PYG{l+m+mi}{7}\PYG{o}{*}\PYG{n}{p44}\PYG{p}{,}\PYG{n}{insert}\PYG{p}{,}\PYG{l+m+mi}{7}\PYG{o}{*}\PYG{n}{p44}\PYG{p}{\PYGZcb{}}  \PYG{c+c1}{\PYGZhy{}\PYGZhy{} \PYGZsh{}: 198 (7*12+30+7*12)}

\PYG{c+c1}{\PYGZhy{}\PYGZhy{} final sequence}
\PYG{k+kd}{local} \PYG{n}{SPS} \PYG{o}{=} \PYG{n}{sequence} \PYG{l+s+s1}{\PYGZsq{}}\PYG{l+s+s1}{SPS}\PYG{l+s+s1}{\PYGZsq{}} \PYG{p}{\PYGZob{}}\PYG{l+m+mi}{6}\PYG{o}{*}\PYG{n}{super}\PYG{p}{\PYGZcb{}}                \PYG{c+c1}{\PYGZhy{}\PYGZhy{} \PYGZsh{} = 1188 (6*198)}

\PYG{c+c1}{\PYGZhy{}\PYGZhy{} check number of elements and length}
\PYG{n+nb}{print}\PYG{p}{(}\PYG{o}{\PYGZsh{}}\PYG{n}{SPS}\PYG{p}{,} \PYG{n}{SPS}\PYG{p}{.}\PYG{n}{l}\PYG{p}{)}  \PYG{c+c1}{\PYGZhy{}\PYGZhy{} display: 1190  0 (no element length provided)}
\end{sphinxVerbatim}


\subsection{Installing elements I}
\label{\detokenize{sequences:installing-elements-i}}
\sphinxAtStartPar
The following example shows how to install elements and subsequences in an empty initial sequence::

\begin{sphinxVerbatim}[commandchars=\\\{\}]
\PYG{k+kd}{local} \PYG{n}{sequence}\PYG{p}{,} \PYG{n}{drift} \PYG{k+kr}{in} \PYG{n}{MAD}\PYG{p}{.}\PYG{n}{element}
\PYG{k+kd}{local} \PYG{n}{seq}   \PYG{o}{=} \PYG{n}{sequence} \PYG{l+s+s2}{\PYGZdq{}}\PYG{l+s+s2}{seq}\PYG{l+s+s2}{\PYGZdq{}} \PYG{p}{\PYGZob{}} \PYG{n}{l}\PYG{o}{=}\PYG{l+m+mi}{16}\PYG{p}{,} \PYG{n}{refer}\PYG{o}{=}\PYG{l+s+s2}{\PYGZdq{}}\PYG{l+s+s2}{entry}\PYG{l+s+s2}{\PYGZdq{}}\PYG{p}{,} \PYG{n}{owner}\PYG{o}{=}\PYG{k+kc}{true} \PYG{p}{\PYGZcb{}}
\PYG{k+kd}{local} \PYG{n}{sseq1} \PYG{o}{=} \PYG{n}{sequence} \PYG{l+s+s2}{\PYGZdq{}}\PYG{l+s+s2}{sseq1}\PYG{l+s+s2}{\PYGZdq{}} \PYG{p}{\PYGZob{}}
\PYG{n}{at}\PYG{o}{=}\PYG{l+m+mi}{5}\PYG{p}{,} \PYG{n}{l}\PYG{o}{=}\PYG{l+m+mi}{6} \PYG{p}{,} \PYG{n}{refpos}\PYG{o}{=}\PYG{l+s+s2}{\PYGZdq{}}\PYG{l+s+s2}{centre}\PYG{l+s+s2}{\PYGZdq{}}\PYG{p}{,} \PYG{n}{refer}\PYG{o}{=}\PYG{l+s+s2}{\PYGZdq{}}\PYG{l+s+s2}{entry}\PYG{l+s+s2}{\PYGZdq{}}\PYG{p}{,}
\PYG{n}{drift} \PYG{l+s+s2}{\PYGZdq{}}\PYG{l+s+s2}{df1\PYGZsq{}}\PYG{l+s+s2}{\PYGZdq{}} \PYG{p}{\PYGZob{}}\PYG{n}{l}\PYG{o}{=}\PYG{l+m+mi}{1}\PYG{p}{,} \PYG{n}{at}\PYG{o}{=\PYGZhy{}}\PYG{l+m+mi}{4}\PYG{p}{,} \PYG{n}{from}\PYG{o}{=}\PYG{l+s+s2}{\PYGZdq{}}\PYG{l+s+s2}{end}\PYG{l+s+s2}{\PYGZdq{}}\PYG{p}{\PYGZcb{}}\PYG{p}{,}
\PYG{n}{drift} \PYG{l+s+s2}{\PYGZdq{}}\PYG{l+s+s2}{df2\PYGZsq{}}\PYG{l+s+s2}{\PYGZdq{}} \PYG{p}{\PYGZob{}}\PYG{n}{l}\PYG{o}{=}\PYG{l+m+mi}{1}\PYG{p}{,} \PYG{n}{at}\PYG{o}{=\PYGZhy{}}\PYG{l+m+mi}{2}\PYG{p}{,} \PYG{n}{from}\PYG{o}{=}\PYG{l+s+s2}{\PYGZdq{}}\PYG{l+s+s2}{end}\PYG{l+s+s2}{\PYGZdq{}}\PYG{p}{\PYGZcb{}}\PYG{p}{,}
\PYG{n}{drift} \PYG{l+s+s2}{\PYGZdq{}}\PYG{l+s+s2}{df3\PYGZsq{}}\PYG{l+s+s2}{\PYGZdq{}} \PYG{p}{\PYGZob{}}     \PYG{n}{at}\PYG{o}{=} \PYG{l+m+mi}{5}            \PYG{p}{\PYGZcb{}}\PYG{p}{,}
\PYG{p}{\PYGZcb{}}
\PYG{k+kd}{local} \PYG{n}{sseq2} \PYG{o}{=} \PYG{n}{sequence} \PYG{l+s+s2}{\PYGZdq{}}\PYG{l+s+s2}{sseq2}\PYG{l+s+s2}{\PYGZdq{}} \PYG{p}{\PYGZob{}}
\PYG{n}{at}\PYG{o}{=}\PYG{l+m+mi}{14}\PYG{p}{,} \PYG{n}{l}\PYG{o}{=}\PYG{l+m+mi}{6}\PYG{p}{,} \PYG{n}{refpos}\PYG{o}{=}\PYG{l+s+s2}{\PYGZdq{}}\PYG{l+s+s2}{exit}\PYG{l+s+s2}{\PYGZdq{}}\PYG{p}{,} \PYG{n}{refer}\PYG{o}{=}\PYG{l+s+s2}{\PYGZdq{}}\PYG{l+s+s2}{entry}\PYG{l+s+s2}{\PYGZdq{}}\PYG{p}{,}
\PYG{n}{drift} \PYG{l+s+s2}{\PYGZdq{}}\PYG{l+s+s2}{df1\PYGZsq{}\PYGZsq{}}\PYG{l+s+s2}{\PYGZdq{}} \PYG{p}{\PYGZob{}} \PYG{n}{l}\PYG{o}{=}\PYG{l+m+mi}{1}\PYG{p}{,} \PYG{n}{at}\PYG{o}{=\PYGZhy{}}\PYG{l+m+mi}{4}\PYG{p}{,} \PYG{n}{from}\PYG{o}{=}\PYG{l+s+s2}{\PYGZdq{}}\PYG{l+s+s2}{end}\PYG{l+s+s2}{\PYGZdq{}}\PYG{p}{\PYGZcb{}}\PYG{p}{,}
\PYG{n}{drift} \PYG{l+s+s2}{\PYGZdq{}}\PYG{l+s+s2}{df2\PYGZsq{}\PYGZsq{}}\PYG{l+s+s2}{\PYGZdq{}} \PYG{p}{\PYGZob{}} \PYG{n}{l}\PYG{o}{=}\PYG{l+m+mi}{1}\PYG{p}{,} \PYG{n}{at}\PYG{o}{=\PYGZhy{}}\PYG{l+m+mi}{2}\PYG{p}{,} \PYG{n}{from}\PYG{o}{=}\PYG{l+s+s2}{\PYGZdq{}}\PYG{l+s+s2}{end}\PYG{l+s+s2}{\PYGZdq{}}\PYG{p}{\PYGZcb{}}\PYG{p}{,}
\PYG{n}{drift} \PYG{l+s+s2}{\PYGZdq{}}\PYG{l+s+s2}{df3\PYGZsq{}\PYGZsq{}}\PYG{l+s+s2}{\PYGZdq{}} \PYG{p}{\PYGZob{}}      \PYG{n}{at}\PYG{o}{=} \PYG{l+m+mi}{5}            \PYG{p}{\PYGZcb{}}\PYG{p}{,}
\PYG{p}{\PYGZcb{}}
\PYG{n}{seq}\PYG{p}{:}\PYG{n}{install} \PYG{p}{\PYGZob{}}
\PYG{n}{drift} \PYG{l+s+s2}{\PYGZdq{}}\PYG{l+s+s2}{df1}\PYG{l+s+s2}{\PYGZdq{}} \PYG{p}{\PYGZob{}}\PYG{n}{l}\PYG{o}{=}\PYG{l+m+mi}{1}\PYG{p}{,} \PYG{n}{at}\PYG{o}{=}\PYG{l+m+mi}{1}\PYG{p}{\PYGZcb{}}\PYG{p}{,}
\PYG{n}{sseq1}\PYG{p}{,} \PYG{n}{sseq2}\PYG{p}{,}
\PYG{n}{drift} \PYG{l+s+s2}{\PYGZdq{}}\PYG{l+s+s2}{df2}\PYG{l+s+s2}{\PYGZdq{}} \PYG{p}{\PYGZob{}}\PYG{n}{l}\PYG{o}{=}\PYG{l+m+mi}{1}\PYG{p}{,} \PYG{n}{at}\PYG{o}{=}\PYG{l+m+mi}{15}\PYG{p}{\PYGZcb{}}\PYG{p}{,}
\PYG{p}{\PYGZcb{}} \PYG{p}{:}\PYG{n}{dumpseq}\PYG{p}{(}\PYG{p}{)}
\end{sphinxVerbatim}

\sphinxAtStartPar
Display:

\begin{sphinxVerbatim}[commandchars=\\\{\}]
sequence: seq, l=16
idx  kind          name       l          dl       spos       upos    uds
001  marker        \PYGZdl{}start*    0.000       0       0.000      0.000   0.000
002  drift         df1        1.000       0       1.000      1.000   0.000
003  drift         df1\PYGZsq{}       1.000       0       4.000      4.000   0.000
004  drift         df2\PYGZsq{}       1.000       0       6.000      6.000   0.000
005  drift         df3\PYGZsq{}       0.000       0       7.000      7.000   0.000
006  drift         df1\PYGZsq{}\PYGZsq{}      1.000       0      10.000     10.000   0.000
007  drift         df2\PYGZsq{}\PYGZsq{}      1.000       0      12.000     12.000   0.000
008  drift         df3\PYGZsq{}\PYGZsq{}      0.000       0      13.000     13.000   0.000
009  drift         df2        1.000       0      15.000     15.000   0.000
010  marker        \PYGZdl{}end       0.000       0      16.000     16.000   0.000
\end{sphinxVerbatim}


\subsection{Installing elements II}
\label{\detokenize{sequences:installing-elements-ii}}
\sphinxAtStartPar
The following more complex example shows how to install elements and subsequences in a sequence using a selection and the packed form for arguments::

\begin{sphinxVerbatim}[commandchars=\\\{\}]
\PYG{k+kd}{local} \PYG{n}{mk}   \PYG{o}{=} \PYG{n}{marker}   \PYG{l+s+s2}{\PYGZdq{}}\PYG{l+s+s2}{mk}\PYG{l+s+s2}{\PYGZdq{}}  \PYG{p}{\PYGZob{}} \PYG{p}{\PYGZcb{}}
\PYG{k+kd}{local} \PYG{n}{seq}  \PYG{o}{=} \PYG{n}{sequence} \PYG{l+s+s2}{\PYGZdq{}}\PYG{l+s+s2}{seq}\PYG{l+s+s2}{\PYGZdq{}} \PYG{p}{\PYGZob{}} \PYG{n}{l} \PYG{o}{=} \PYG{l+m+mi}{10}\PYG{p}{,} \PYG{n}{refer}\PYG{o}{=}\PYG{l+s+s2}{\PYGZdq{}}\PYG{l+s+s2}{entry}\PYG{l+s+s2}{\PYGZdq{}}\PYG{p}{,}
\PYG{n}{mk} \PYG{l+s+s2}{\PYGZdq{}}\PYG{l+s+s2}{mk1}\PYG{l+s+s2}{\PYGZdq{}} \PYG{p}{\PYGZob{}} \PYG{n}{at} \PYG{o}{=} \PYG{l+m+mi}{2} \PYG{p}{\PYGZcb{}}\PYG{p}{,}
\PYG{n}{mk} \PYG{l+s+s2}{\PYGZdq{}}\PYG{l+s+s2}{mk2}\PYG{l+s+s2}{\PYGZdq{}} \PYG{p}{\PYGZob{}} \PYG{n}{at} \PYG{o}{=} \PYG{l+m+mi}{4} \PYG{p}{\PYGZcb{}}\PYG{p}{,}
\PYG{n}{mk} \PYG{l+s+s2}{\PYGZdq{}}\PYG{l+s+s2}{mk3}\PYG{l+s+s2}{\PYGZdq{}} \PYG{p}{\PYGZob{}} \PYG{n}{at} \PYG{o}{=} \PYG{l+m+mi}{8} \PYG{p}{\PYGZcb{}}\PYG{p}{,}
\PYG{p}{\PYGZcb{}}
\PYG{k+kd}{local} \PYG{n}{sseq} \PYG{o}{=} \PYG{n}{sequence} \PYG{l+s+s2}{\PYGZdq{}}\PYG{l+s+s2}{sseq}\PYG{l+s+s2}{\PYGZdq{}} \PYG{p}{\PYGZob{}} \PYG{n}{l} \PYG{o}{=} \PYG{l+m+mi}{3} \PYG{p}{,} \PYG{n}{at} \PYG{o}{=} \PYG{l+m+mi}{5}\PYG{p}{,} \PYG{n}{refer}\PYG{o}{=}\PYG{l+s+s2}{\PYGZdq{}}\PYG{l+s+s2}{entry}\PYG{l+s+s2}{\PYGZdq{}}\PYG{p}{,}
\PYG{n}{drift} \PYG{l+s+s2}{\PYGZdq{}}\PYG{l+s+s2}{df1\PYGZsq{}}\PYG{l+s+s2}{\PYGZdq{}} \PYG{p}{\PYGZob{}} \PYG{n}{l} \PYG{o}{=} \PYG{l+m+mi}{1}\PYG{p}{,} \PYG{n}{at} \PYG{o}{=} \PYG{l+m+mi}{0} \PYG{p}{\PYGZcb{}}\PYG{p}{,}
\PYG{n}{drift} \PYG{l+s+s2}{\PYGZdq{}}\PYG{l+s+s2}{df2\PYGZsq{}}\PYG{l+s+s2}{\PYGZdq{}} \PYG{p}{\PYGZob{}} \PYG{n}{l} \PYG{o}{=} \PYG{l+m+mi}{1}\PYG{p}{,} \PYG{n}{at} \PYG{o}{=} \PYG{l+m+mi}{1} \PYG{p}{\PYGZcb{}}\PYG{p}{,}
\PYG{n}{drift} \PYG{l+s+s2}{\PYGZdq{}}\PYG{l+s+s2}{df3\PYGZsq{}}\PYG{l+s+s2}{\PYGZdq{}} \PYG{p}{\PYGZob{}} \PYG{n}{l} \PYG{o}{=} \PYG{l+m+mi}{1}\PYG{p}{,} \PYG{n}{at} \PYG{o}{=} \PYG{l+m+mi}{2} \PYG{p}{\PYGZcb{}}\PYG{p}{,}
\PYG{p}{\PYGZcb{}}
\PYG{n}{seq}\PYG{p}{:}\PYG{n}{install} \PYG{p}{\PYGZob{}}
\PYG{n}{class}    \PYG{o}{=} \PYG{n}{mk}\PYG{p}{,}
\PYG{n}{elements} \PYG{o}{=} \PYG{p}{\PYGZob{}}
   \PYG{n}{drift} \PYG{l+s+s2}{\PYGZdq{}}\PYG{l+s+s2}{df1}\PYG{l+s+s2}{\PYGZdq{}} \PYG{p}{\PYGZob{}} \PYG{n}{l} \PYG{o}{=} \PYG{l+m+mf}{0.1}\PYG{p}{,} \PYG{n}{at} \PYG{o}{=} \PYG{l+m+mf}{0.1}\PYG{p}{,} \PYG{n}{from}\PYG{o}{=}\PYG{l+s+s2}{\PYGZdq{}}\PYG{l+s+s2}{selected}\PYG{l+s+s2}{\PYGZdq{}} \PYG{p}{\PYGZcb{}}\PYG{p}{,}
   \PYG{n}{drift} \PYG{l+s+s2}{\PYGZdq{}}\PYG{l+s+s2}{df2}\PYG{l+s+s2}{\PYGZdq{}} \PYG{p}{\PYGZob{}} \PYG{n}{l} \PYG{o}{=} \PYG{l+m+mf}{0.1}\PYG{p}{,} \PYG{n}{at} \PYG{o}{=} \PYG{l+m+mf}{0.2}\PYG{p}{,} \PYG{n}{from}\PYG{o}{=}\PYG{l+s+s2}{\PYGZdq{}}\PYG{l+s+s2}{selected}\PYG{l+s+s2}{\PYGZdq{}} \PYG{p}{\PYGZcb{}}\PYG{p}{,}
   \PYG{n}{drift} \PYG{l+s+s2}{\PYGZdq{}}\PYG{l+s+s2}{df3}\PYG{l+s+s2}{\PYGZdq{}} \PYG{p}{\PYGZob{}} \PYG{n}{l} \PYG{o}{=} \PYG{l+m+mf}{0.1}\PYG{p}{,} \PYG{n}{at} \PYG{o}{=} \PYG{l+m+mf}{0.3}\PYG{p}{,} \PYG{n}{from}\PYG{o}{=}\PYG{l+s+s2}{\PYGZdq{}}\PYG{l+s+s2}{selected}\PYG{l+s+s2}{\PYGZdq{}} \PYG{p}{\PYGZcb{}}\PYG{p}{,}
   \PYG{n}{sseq}\PYG{p}{,}
   \PYG{n}{drift} \PYG{l+s+s2}{\PYGZdq{}}\PYG{l+s+s2}{df4}\PYG{l+s+s2}{\PYGZdq{}} \PYG{p}{\PYGZob{}} \PYG{n}{l} \PYG{o}{=} \PYG{l+m+mi}{1}\PYG{p}{,} \PYG{n}{at} \PYG{o}{=} \PYG{l+m+mi}{9} \PYG{p}{\PYGZcb{}}\PYG{p}{,}
\PYG{p}{\PYGZcb{}}
\PYG{p}{\PYGZcb{}}

\PYG{n}{seq}\PYG{p}{:}\PYG{n}{dumpseq}\PYG{p}{(}\PYG{p}{)}
\end{sphinxVerbatim}

\begin{sphinxVerbatim}[commandchars=\\\{\}]
sequence: seq, l=10
idx  kind          name      l          dl       spos       upos    uds
001  marker        \PYGZdl{}start    0.000       0       0.000      0.000   0.000
002  marker        mk1       0.000       0       2.000      2.000   0.000
003  drift         df1       0.100       0       2.100      2.100   0.000
004  drift         df2       0.100       0       2.200      2.200   0.000
005  drift         df3       0.100       0       2.300      2.300   0.000
006  marker        mk2       0.000       0       4.000      4.000   0.000
007  drift         df1       0.100       0       4.100      4.100   0.000
008  drift         df2       0.100       0       4.200      4.200   0.000
009  drift         df3       0.100       0       4.300      4.300   0.000
010  drift         df1\PYGZsq{}      1.000       0       5.000      5.000   0.000
011  drift         df2\PYGZsq{}      1.000       0       6.000      6.000   0.000
012  drift         df3\PYGZsq{}      1.000       0       7.000      7.000   0.000
013  marker        mk3       0.000       0       8.000      8.000   0.000
014  drift         df1       0.100       0       8.100      8.100   0.000
015  drift         df2       0.100       0       8.200      8.200   0.000
016  drift         df3       0.100       0       8.300      8.300   0.000
017  drift         df4       1.000       0       9.000      9.000   0.000
018  marker        \PYGZdl{}end      0.000       0      10.000     10.000   0.000
\end{sphinxVerbatim}


\section{Random Maths}
\label{\detokenize{sequences:random-maths}}\begin{equation*}
\begin{split}\nabla \cdot \textbf{E} = \frac{\rho}{\epsilon_0}\\\\
\nabla \cdot \textbf{B} = 0\\\\
\nabla \times \textbf{E} =- \frac{\partial \textbf{B}}{\partial t}\\\\
\nabla \times \textbf{B} = \mu_0 \textbf{J} + \mu_0 \epsilon_0 \frac{\partial \textbf{E}}{\partial t}\end{split}
\end{equation*}
\sphinxstepscope


\chapter{Types}
\label{\detokenize{types:types}}\label{\detokenize{types::doc}}
\sphinxAtStartPar
Just here to be a link

\sphinxstepscope


\chapter{Generic Utilities}
\label{\detokenize{utility:generic-utilities}}\label{\detokenize{utility::doc}}
\sphinxAtStartPar
Just here to be linked to

\sphinxstepscope

\index{elementary mathematical and physical constants@\spxentry{elementary mathematical and physical constants}}\ignorespaces 

\chapter{Elementary Constants}
\label{\detokenize{constants:elementary-constants}}\label{\detokenize{constants:index-0}}\label{\detokenize{constants::doc}}
\sphinxAtStartPar
This chapter describes basic mathematical and physiscal constants provided by the module \sphinxcode{\sphinxupquote{constant}}.


\section{Mathematical Constants}
\label{\detokenize{constants:mathematical-constants}}
\sphinxAtStartPar
This section describes basic mathematical constants uniquely defined as macros in the C header \sphinxcode{\sphinxupquote{mad\_cst.h}} and available from C and MAD modules. If these mathematical constants are already provided by the system libraries, they are used instead of the local definitions.


\begin{savenotes}\sphinxattablestart
\centering
\begin{tabular}[t]{|*{4}{\X{1}{4}|}}
\hline
\sphinxstyletheadfamily 
\sphinxAtStartPar
MAD constants
&\sphinxstyletheadfamily 
\sphinxAtStartPar
C macros
&\sphinxstyletheadfamily 
\sphinxAtStartPar
C constants
&\sphinxstyletheadfamily 
\sphinxAtStartPar
Values
\\
\hline
\sphinxAtStartPar
\sphinxcode{\sphinxupquote{eps}}
&
\sphinxAtStartPar
\sphinxcode{\sphinxupquote{DBL\_EPSILON}}
&
\sphinxAtStartPar
\sphinxcode{\sphinxupquote{mad\_cst\_EPS}}
&
\sphinxAtStartPar
Smallest representable increment near one
\\
\hline
\sphinxAtStartPar
\sphinxcode{\sphinxupquote{tiny}}
&
\sphinxAtStartPar
\sphinxcode{\sphinxupquote{DBL\_MIN}}
&
\sphinxAtStartPar
\sphinxcode{\sphinxupquote{mad\_cst\_TINY}}
&
\sphinxAtStartPar
Smallest representable number
\\
\hline
\sphinxAtStartPar
\sphinxcode{\sphinxupquote{huge}}
&
\sphinxAtStartPar
\sphinxcode{\sphinxupquote{DBL\_MAX}}
&
\sphinxAtStartPar
\sphinxcode{\sphinxupquote{mad\_cst\_HUGE}}
&
\sphinxAtStartPar
Largest representable number
\\
\hline
\sphinxAtStartPar
\sphinxcode{\sphinxupquote{inf}}
&
\sphinxAtStartPar
\sphinxcode{\sphinxupquote{INFINITY}}
&
\sphinxAtStartPar
\sphinxcode{\sphinxupquote{mad\_cst\_INF}}
&
\sphinxAtStartPar
Positive infinity, \(1/0\)
\\
\hline
\sphinxAtStartPar
\sphinxcode{\sphinxupquote{nan}}
&\begin{itemize}
\item {} 
\end{itemize}
&\begin{itemize}
\item {} 
\end{itemize}
&
\sphinxAtStartPar
Canonical NaN, \(0/0\)
\\
\hline
\sphinxAtStartPar
\sphinxcode{\sphinxupquote{e}}
&
\sphinxAtStartPar
\sphinxcode{\sphinxupquote{M\_E}}
&
\sphinxAtStartPar
\sphinxcode{\sphinxupquote{mad\_cst\_E}}
&
\sphinxAtStartPar
\(e, \exp(1)\)
\\
\hline
\sphinxAtStartPar
\sphinxcode{\sphinxupquote{log2e}}
&
\sphinxAtStartPar
\sphinxcode{\sphinxupquote{M\_LOG2E}}
&
\sphinxAtStartPar
\sphinxcode{\sphinxupquote{mad\_cst\_LOG2E}}
&
\sphinxAtStartPar
\(\log_2(e)\)
\\
\hline
\sphinxAtStartPar
\sphinxcode{\sphinxupquote{log10e}}
&
\sphinxAtStartPar
\sphinxcode{\sphinxupquote{M\_LOG10E}}
&
\sphinxAtStartPar
\sphinxcode{\sphinxupquote{mad\_cst\_LOG10E}}
&
\sphinxAtStartPar
\(\log_{10}(e)\)
\\
\hline
\sphinxAtStartPar
\sphinxcode{\sphinxupquote{ln2}}
&
\sphinxAtStartPar
\sphinxcode{\sphinxupquote{M\_LN2}}
&
\sphinxAtStartPar
\sphinxcode{\sphinxupquote{mad\_cst\_LN2}}
&
\sphinxAtStartPar
\(\ln(2)\)
\\
\hline
\sphinxAtStartPar
\sphinxcode{\sphinxupquote{ln10}}
&
\sphinxAtStartPar
\sphinxcode{\sphinxupquote{M\_LN10}}
&
\sphinxAtStartPar
\sphinxcode{\sphinxupquote{mad\_cst\_LN10}}
&
\sphinxAtStartPar
\(\ln(10)\)
\\
\hline
\sphinxAtStartPar
\sphinxcode{\sphinxupquote{lnpi}}
&
\sphinxAtStartPar
\sphinxcode{\sphinxupquote{M\_LNPI}}
&
\sphinxAtStartPar
\sphinxcode{\sphinxupquote{mad\_cst\_LNPI}}
&
\sphinxAtStartPar
\(\ln(\pi)\)
\\
\hline
\sphinxAtStartPar
\sphinxcode{\sphinxupquote{pi}}
&
\sphinxAtStartPar
\sphinxcode{\sphinxupquote{M\_PI}}
&
\sphinxAtStartPar
\sphinxcode{\sphinxupquote{mad\_cst\_PI}}
&
\sphinxAtStartPar
\(\pi\)
\\
\hline
\sphinxAtStartPar
\sphinxcode{\sphinxupquote{twopi}}
&
\sphinxAtStartPar
\sphinxcode{\sphinxupquote{M\_2PI}}
&
\sphinxAtStartPar
\sphinxcode{\sphinxupquote{mad\_cst\_2PI}}
&
\sphinxAtStartPar
\(2\pi\)
\\
\hline
\sphinxAtStartPar
\sphinxcode{\sphinxupquote{pi\_2}}
&
\sphinxAtStartPar
\sphinxcode{\sphinxupquote{M\_PI\_2}}
&
\sphinxAtStartPar
\sphinxcode{\sphinxupquote{mad\_cst\_PI\_2}}
&
\sphinxAtStartPar
\(\pi/2\)
\\
\hline
\sphinxAtStartPar
\sphinxcode{\sphinxupquote{pi\_4}}
&
\sphinxAtStartPar
\sphinxcode{\sphinxupquote{M\_PI\_4}}
&
\sphinxAtStartPar
\sphinxcode{\sphinxupquote{mad\_cst\_PI\_4}}
&
\sphinxAtStartPar
\(\pi/4\)
\\
\hline
\sphinxAtStartPar
\sphinxcode{\sphinxupquote{one\_pi}}
&
\sphinxAtStartPar
\sphinxcode{\sphinxupquote{M\_1\_PI}}
&
\sphinxAtStartPar
\sphinxcode{\sphinxupquote{mad\_cst\_1\_PI}}
&
\sphinxAtStartPar
\(1/\pi\)
\\
\hline
\sphinxAtStartPar
\sphinxcode{\sphinxupquote{two\_pi}}
&
\sphinxAtStartPar
\sphinxcode{\sphinxupquote{M\_2\_PI}}
&
\sphinxAtStartPar
\sphinxcode{\sphinxupquote{mad\_cst\_2\_PI}}
&
\sphinxAtStartPar
\(2/\pi\)
\\
\hline
\sphinxAtStartPar
\sphinxcode{\sphinxupquote{sqrt2}}
&
\sphinxAtStartPar
\sphinxcode{\sphinxupquote{M\_SQRT2}}
&
\sphinxAtStartPar
\sphinxcode{\sphinxupquote{mad\_cst\_SQRT2}}
&
\sphinxAtStartPar
\(\sqrt 2\)
\\
\hline
\sphinxAtStartPar
\sphinxcode{\sphinxupquote{sqrt3}}
&
\sphinxAtStartPar
\sphinxcode{\sphinxupquote{M\_SQRT3}}
&
\sphinxAtStartPar
\sphinxcode{\sphinxupquote{mad\_cst\_SQRT3}}
&
\sphinxAtStartPar
\(\sqrt 3\)
\\
\hline
\sphinxAtStartPar
\sphinxcode{\sphinxupquote{sqrtpi}}
&
\sphinxAtStartPar
\sphinxcode{\sphinxupquote{M\_SQRTPI}}
&
\sphinxAtStartPar
\sphinxcode{\sphinxupquote{mad\_cst\_SQRTPI}}
&
\sphinxAtStartPar
\(\sqrt{\pi}\)
\\
\hline
\sphinxAtStartPar
\sphinxcode{\sphinxupquote{sqrt1\_2}}
&
\sphinxAtStartPar
\sphinxcode{\sphinxupquote{M\_SQRT1\_2}}
&
\sphinxAtStartPar
\sphinxcode{\sphinxupquote{mad\_cst\_SQRT1\_2}}
&
\sphinxAtStartPar
\(\sqrt{1/2}\)
\\
\hline
\sphinxAtStartPar
\sphinxcode{\sphinxupquote{sqrt1\_3}}
&
\sphinxAtStartPar
\sphinxcode{\sphinxupquote{M\_SQRT1\_3}}
&
\sphinxAtStartPar
\sphinxcode{\sphinxupquote{mad\_cst\_SQRT1\_3}}
&
\sphinxAtStartPar
\(\sqrt{1/3}\)
\\
\hline
\sphinxAtStartPar
\sphinxcode{\sphinxupquote{one\_sqrtpi}}
&
\sphinxAtStartPar
\sphinxcode{\sphinxupquote{M\_1\_SQRTPI}}
&
\sphinxAtStartPar
\sphinxcode{\sphinxupquote{mad\_cst\_1\_SQRTPI}}
&
\sphinxAtStartPar
\(1/\sqrt{\pi}\)
\\
\hline
\sphinxAtStartPar
\sphinxcode{\sphinxupquote{two\_sqrtpi}}
&
\sphinxAtStartPar
\sphinxcode{\sphinxupquote{M\_2\_SQRTPI}}
&
\sphinxAtStartPar
\sphinxcode{\sphinxupquote{mad\_cst\_2\_SQRTPI}}
&
\sphinxAtStartPar
\(2/\sqrt{\pi}\)
\\
\hline
\sphinxAtStartPar
\sphinxcode{\sphinxupquote{raddeg}}
&
\sphinxAtStartPar
\sphinxcode{\sphinxupquote{M\_RADDEG}}
&
\sphinxAtStartPar
\sphinxcode{\sphinxupquote{mad\_cst\_RADDEG}}
&
\sphinxAtStartPar
\(180/\pi\)
\\
\hline
\sphinxAtStartPar
\sphinxcode{\sphinxupquote{degrad}}
&
\sphinxAtStartPar
\sphinxcode{\sphinxupquote{M\_DEGRAD}}
&
\sphinxAtStartPar
\sphinxcode{\sphinxupquote{mad\_cst\_DEGRAD}}
&
\sphinxAtStartPar
\(\pi/180\)
\\
\hline
\end{tabular}
\par
\sphinxattableend\end{savenotes}

\index{mathematical constants@\spxentry{mathematical constants}}\ignorespaces 

\section{Physical Constants}
\label{\detokenize{constants:physical-constants}}\label{\detokenize{constants:index-1}}
\sphinxAtStartPar
This section describes basic physical constants uniquely defined as macros in the C header \sphinxcode{\sphinxupquote{mad\_cst.h}} and available from C and MAD modules.


\begin{savenotes}\sphinxattablestart
\centering
\begin{tabulary}{\linewidth}[t]{|T|T|T|T|}
\hline
\sphinxstyletheadfamily 
\sphinxAtStartPar
MAD constants
&\sphinxstyletheadfamily 
\sphinxAtStartPar
C macros
&\sphinxstyletheadfamily 
\sphinxAtStartPar
C constants
&\sphinxstyletheadfamily 
\sphinxAtStartPar
Values
\\
\hline
\sphinxAtStartPar
\sphinxcode{\sphinxupquote{minlen}}
&
\sphinxAtStartPar
\sphinxcode{\sphinxupquote{P\_MINLEN}}
&
\sphinxAtStartPar
\sphinxcode{\sphinxupquote{mad\_cst\_MINLEN}}
&
\sphinxAtStartPar
Minimum length tolerance, \(10^-10\) in \sphinxstylestrong{{[}m{]}}
\\
\hline
\sphinxAtStartPar
\sphinxcode{\sphinxupquote{minang}}
&
\sphinxAtStartPar
\sphinxcode{\sphinxupquote{P\_MINANG}}
&
\sphinxAtStartPar
\sphinxcode{\sphinxupquote{mad\_cst\_MINANG}}
&
\sphinxAtStartPar
Minimum angle tolerance, \(10^-10\) in \sphinxstylestrong{{[}m\textasciicircum{}\{\sphinxhyphen{}1\}{]}}
\\
\hline
\sphinxAtStartPar
\sphinxcode{\sphinxupquote{minstr}}
&
\sphinxAtStartPar
\sphinxcode{\sphinxupquote{P\_MINSTR}}
&
\sphinxAtStartPar
\sphinxcode{\sphinxupquote{mad\_cst\_MINSTR}}
&
\sphinxAtStartPar
Minimum strength tolerance, \(10^-10\) in \sphinxstylestrong{{[}rad{]}}
\\
\hline
\end{tabulary}
\par
\sphinxattableend\end{savenotes}

\sphinxAtStartPar
The following table lists some physical constants from the \sphinxhref{https://physics.nist.gov/cuu/pdf/wall\_2018.pdf}{CODATA 2018} sheet.


\begin{savenotes}\sphinxattablestart
\centering
\begin{tabulary}{\linewidth}[t]{|T|T|T|T|}
\hline
\sphinxstyletheadfamily 
\sphinxAtStartPar
MAD constants
&\sphinxstyletheadfamily 
\sphinxAtStartPar
C macros
&\sphinxstyletheadfamily 
\sphinxAtStartPar
C constants
&\sphinxstyletheadfamily 
\sphinxAtStartPar
Values
\\
\hline
\sphinxAtStartPar
\sphinxcode{\sphinxupquote{clight}}
&
\sphinxAtStartPar
\sphinxcode{\sphinxupquote{P\_CLIGHT}}
&
\sphinxAtStartPar
\sphinxcode{\sphinxupquote{mad\_cst\_CLIGHT}}
&
\sphinxAtStartPar
Speed of light, \(c\) in \sphinxstylestrong{{[}m/s{]}}
\\
\hline
\sphinxAtStartPar
\sphinxcode{\sphinxupquote{mu0}}
&
\sphinxAtStartPar
\sphinxcode{\sphinxupquote{P\_MU0}}
&
\sphinxAtStartPar
\sphinxcode{\sphinxupquote{mad\_cst\_MU0}}
&
\sphinxAtStartPar
Permeability of vacuum, \(\mu_0\) in \sphinxstylestrong{{[}T.m/A{]}}
\\
\hline
\sphinxAtStartPar
\sphinxcode{\sphinxupquote{epsilon0}}
&
\sphinxAtStartPar
\sphinxcode{\sphinxupquote{P\_EPSILON0}}
&
\sphinxAtStartPar
\sphinxcode{\sphinxupquote{mad\_cst\_EPSILON0}}
&
\sphinxAtStartPar
Permittivity of vacuum, \(\epsilon_0\) in \sphinxstylestrong{{[}F/m{]}}
\\
\hline
\sphinxAtStartPar
\sphinxcode{\sphinxupquote{qelect}}
&
\sphinxAtStartPar
\sphinxcode{\sphinxupquote{P\_QELECT}}
&
\sphinxAtStartPar
\sphinxcode{\sphinxupquote{mad\_cst\_QELECT}}
&
\sphinxAtStartPar
Elementary electric charge, \(e\) in \sphinxstylestrong{{[}C{]}}
\\
\hline
\sphinxAtStartPar
\sphinxcode{\sphinxupquote{hbar}}
&
\sphinxAtStartPar
\sphinxcode{\sphinxupquote{P\_HBAR}}
&
\sphinxAtStartPar
\sphinxcode{\sphinxupquote{mad\_cst\_HBAR}}
&
\sphinxAtStartPar
Reduced Plack’s constant, \(\hbar\) in \sphinxstylestrong{{[}GeV.s{]}}
\\
\hline
\sphinxAtStartPar
\sphinxcode{\sphinxupquote{amass}}
&
\sphinxAtStartPar
\sphinxcode{\sphinxupquote{P\_AMASS}}
&
\sphinxAtStartPar
\sphinxcode{\sphinxupquote{mad\_cst\_AMASS}}
&
\sphinxAtStartPar
Unified atomic mass, \(m_u c^2\) in \sphinxstylestrong{{[}GeV{]}}
\\
\hline
\sphinxAtStartPar
\sphinxcode{\sphinxupquote{emass}}
&
\sphinxAtStartPar
\sphinxcode{\sphinxupquote{P\_EMASS}}
&
\sphinxAtStartPar
\sphinxcode{\sphinxupquote{mad\_cst\_EMASS}}
&
\sphinxAtStartPar
Electron mass, \(m_e c^2\) in \sphinxstylestrong{{[}GeV{]}}
\\
\hline
\sphinxAtStartPar
\sphinxcode{\sphinxupquote{pmass}}
&
\sphinxAtStartPar
\sphinxcode{\sphinxupquote{P\_PMASS}}
&
\sphinxAtStartPar
\sphinxcode{\sphinxupquote{mad\_cst\_PMASS}}
&
\sphinxAtStartPar
Proton mass, \(m_p c^2\) in \sphinxstylestrong{{[}GeV{]}}
\\
\hline
\sphinxAtStartPar
\sphinxcode{\sphinxupquote{nmass}}
&
\sphinxAtStartPar
\sphinxcode{\sphinxupquote{P\_NMASS}}
&
\sphinxAtStartPar
\sphinxcode{\sphinxupquote{mad\_cst\_NMASS}}
&
\sphinxAtStartPar
Neutron mass, \(m_n c^2\) in \sphinxstylestrong{{[}GeV{]}}
\\
\hline
\sphinxAtStartPar
\sphinxcode{\sphinxupquote{mumass}}
&
\sphinxAtStartPar
\sphinxcode{\sphinxupquote{P\_MUMASS}}
&
\sphinxAtStartPar
\sphinxcode{\sphinxupquote{mad\_cst\_MUMASS}}
&
\sphinxAtStartPar
Muon mass, \(m_{\mu} c^2\) in \sphinxstylestrong{{[}GeV{]}}
\\
\hline
\sphinxAtStartPar
\sphinxcode{\sphinxupquote{deumass}}
&
\sphinxAtStartPar
\sphinxcode{\sphinxupquote{P\_DEUMASS}}
&
\sphinxAtStartPar
\sphinxcode{\sphinxupquote{mad\_cst\_DEUMASS}}
&
\sphinxAtStartPar
Deuteron mass, \(m_d c^2\) in \sphinxstylestrong{{[}GeV{]}}
\\
\hline
\sphinxAtStartPar
\sphinxcode{\sphinxupquote{eradius}}
&
\sphinxAtStartPar
\sphinxcode{\sphinxupquote{P\_ERADIUS}}
&
\sphinxAtStartPar
\sphinxcode{\sphinxupquote{mad\_cst\_ERADIUS}}
&
\sphinxAtStartPar
Classical electron radius, \(r_e\) in \sphinxstylestrong{{[}m{]}}
\\
\hline
\sphinxAtStartPar
\sphinxcode{\sphinxupquote{alphaem}}
&
\sphinxAtStartPar
\sphinxcode{\sphinxupquote{P\_ALPHAEM}}
&
\sphinxAtStartPar
\sphinxcode{\sphinxupquote{mad\_cst\_ALPHAEM}}
&
\sphinxAtStartPar
Fine\sphinxhyphen{}structure constant, \(\alpha\)
\\
\hline
\end{tabulary}
\par
\sphinxattableend\end{savenotes}

\index{physical constants@\spxentry{physical constants}}\index{CODATA@\spxentry{CODATA}}\ignorespaces 
\sphinxstepscope


\chapter{Elements}
\label{\detokenize{elements:elements}}\label{\detokenize{elements::doc}}
\sphinxAtStartPar
Just a link


\section{Misalignment}
\label{\detokenize{elements:misalignment}}\label{\detokenize{elements:elm-misalign}}
\sphinxstepscope


\chapter{Track}
\label{\detokenize{track:track}}\label{\detokenize{track::doc}}
\sphinxAtStartPar
Just a link

\sphinxstepscope


\chapter{Survey}
\label{\detokenize{survey:survey}}\label{\detokenize{survey::doc}}
\sphinxAtStartPar
Just a link

\sphinxstepscope

\index{elementary mathematical functions@\spxentry{elementary mathematical functions}}\ignorespaces 

\chapter{Elementary functions}
\label{\detokenize{functions:elementary-functions}}\label{\detokenize{functions:index-0}}\label{\detokenize{functions::doc}}
\sphinxAtStartPar
This chapter describes elementary functions provided by the module \sphinxcode{\sphinxupquote{gmath}}. This module extends the standard module \sphinxcode{\sphinxupquote{math}} with \sphinxstyleemphasis{generic} functions working on any type that implements the methods with the same name. For example, the code \sphinxcode{\sphinxupquote{gmath.sin(a)}} will call \sphinxcode{\sphinxupquote{math.sin(a)}} if \sphinxcode{\sphinxupquote{a}} is a \sphinxstyleemphasis{number}, otherwise it will calling the method \sphinxcode{\sphinxupquote{a:sin()}}, i.e. delegate the call to \sphinxcode{\sphinxupquote{a}}. This is how MAD\sphinxhyphen{}NG handles few types like \sphinxstyleemphasis{numbers}, \sphinxstyleemphasis{complex} number, \sphinxstyleemphasis{matrix} and \sphinxstyleemphasis{TPSA} within a single code.


\section{Generic Operators}
\label{\detokenize{functions:generic-operators}}
\sphinxAtStartPar
Generic operators are named functions that rely on associated operators, which themselves can be redefined by their associated metamethods.


\begin{savenotes}\sphinxattablestart
\centering
\begin{tabular}[t]{|*{3}{\X{1}{3}|}}
\hline
\sphinxstyletheadfamily 
\sphinxAtStartPar
Operators
&\sphinxstyletheadfamily 
\sphinxAtStartPar
Return values
&\sphinxstyletheadfamily 
\sphinxAtStartPar
Metamethods
\\
\hline
\sphinxAtStartPar
\sphinxcode{\sphinxupquote{unm(x)}}
&
\sphinxAtStartPar
\sphinxcode{\sphinxupquote{\sphinxhyphen{}x}}
&
\sphinxAtStartPar
\_\_unm(x,\_)
\\
\hline
\sphinxAtStartPar
\sphinxcode{\sphinxupquote{add(x,y)}}
&
\sphinxAtStartPar
\sphinxcode{\sphinxupquote{x + y}}
&
\sphinxAtStartPar
\_\_add(x,y)
\\
\hline
\sphinxAtStartPar
\sphinxcode{\sphinxupquote{sub(x,y)}}
&
\sphinxAtStartPar
\sphinxcode{\sphinxupquote{x \sphinxhyphen{} y}}
&
\sphinxAtStartPar
\_\_sub(x,y)
\\
\hline
\sphinxAtStartPar
\sphinxcode{\sphinxupquote{mul(x,y)}}
&
\sphinxAtStartPar
\sphinxcode{\sphinxupquote{x * y}}
&
\sphinxAtStartPar
\_\_mul(x,y)
\\
\hline
\sphinxAtStartPar
\sphinxcode{\sphinxupquote{div(x,y)}}
&
\sphinxAtStartPar
\sphinxcode{\sphinxupquote{x / y}}
&
\sphinxAtStartPar
\_\_div(x,y)
\\
\hline
\sphinxAtStartPar
\sphinxcode{\sphinxupquote{mod(x,y)}}
&
\sphinxAtStartPar
\sphinxcode{\sphinxupquote{x \% y}}
&
\sphinxAtStartPar
\_\_mod(x,y)
\\
\hline
\sphinxAtStartPar
\sphinxcode{\sphinxupquote{pow(x,y)}}
&
\sphinxAtStartPar
\sphinxcode{\sphinxupquote{x \textasciicircum{} y}}
&
\sphinxAtStartPar
\_\_pow(x,y)
\\
\hline
\sphinxAtStartPar
\sphinxcode{\sphinxupquote{sqr(x)}}
&
\sphinxAtStartPar
\sphinxcode{\sphinxupquote{x * x}}
&\begin{itemize}
\item {} 
\end{itemize}
\\
\hline
\sphinxAtStartPar
\sphinxcode{\sphinxupquote{inv(x)}}
&
\sphinxAtStartPar
\sphinxcode{\sphinxupquote{1 / x}}
&\begin{itemize}
\item {} 
\end{itemize}
\\
\hline
\sphinxAtStartPar
\sphinxcode{\sphinxupquote{emul(x,y,r\_)}}
&
\sphinxAtStartPar
\sphinxcode{\sphinxupquote{x .* y}}
&
\sphinxAtStartPar
\_\_emul(x,y,r\_)
\\
\hline
\sphinxAtStartPar
\sphinxcode{\sphinxupquote{ediv(x,y,r\_)}}
&
\sphinxAtStartPar
\sphinxcode{\sphinxupquote{x ./ y}}
&
\sphinxAtStartPar
\_\_ediv(x,y,r\_)
\\
\hline
\sphinxAtStartPar
\sphinxcode{\sphinxupquote{emod(x,y,r\_)}}
&
\sphinxAtStartPar
\sphinxcode{\sphinxupquote{x .\% y}}
&
\sphinxAtStartPar
\_\_emod(x,y,r\_)
\\
\hline
\sphinxAtStartPar
\sphinxcode{\sphinxupquote{epow(x,y,r\_)}}
&
\sphinxAtStartPar
\sphinxcode{\sphinxupquote{x .\textasciicircum{} y}}
&
\sphinxAtStartPar
\_\_epow(x,y,r\_)
\\
\hline
\end{tabular}
\par
\sphinxattableend\end{savenotes}


\section{Generic Functions (real\sphinxhyphen{}like)}
\label{\detokenize{functions:generic-functions-real-like}}
\sphinxAtStartPar
Real\sphinxhyphen{}like generic functions forward the call to the method of the same name from the first argument when the later is not a \sphinxstyleemphasis{number}.


\begin{savenotes}\sphinxatlongtablestart\begin{longtable}[c]{|l|l|l|}
\hline
\sphinxstyletheadfamily 
\sphinxAtStartPar
Functions
&\sphinxstyletheadfamily 
\sphinxAtStartPar
Return values
&\sphinxstyletheadfamily 
\sphinxAtStartPar
C functions
\\
\hline
\endfirsthead

\multicolumn{3}{c}%
{\makebox[0pt]{\sphinxtablecontinued{\tablename\ \thetable{} \textendash{} continued from previous page}}}\\
\hline
\sphinxstyletheadfamily 
\sphinxAtStartPar
Functions
&\sphinxstyletheadfamily 
\sphinxAtStartPar
Return values
&\sphinxstyletheadfamily 
\sphinxAtStartPar
C functions
\\
\hline
\endhead

\hline
\multicolumn{3}{r}{\makebox[0pt][r]{\sphinxtablecontinued{continues on next page}}}\\
\endfoot

\endlastfoot

\sphinxAtStartPar
\sphinxcode{\sphinxupquote{abs    (x)}}
&
\sphinxAtStartPar
\(|x|\)
&\\
\hline
\sphinxAtStartPar
\sphinxcode{\sphinxupquote{acos   (x)}}
&
\sphinxAtStartPar
\(\cos^{-1}(x)\)
&\\
\hline
\sphinxAtStartPar
\sphinxcode{\sphinxupquote{acosh  (x)}}
&
\sphinxAtStartPar
\(\cosh^{-1}(x)\)
&
\sphinxAtStartPar
\sphinxcode{\sphinxupquote{acosh()}}
\\
\hline
\sphinxAtStartPar
\sphinxcode{\sphinxupquote{acot   (x)}}
&
\sphinxAtStartPar
\(\cot^{-1}(x)\)
&\\
\hline
\sphinxAtStartPar
\sphinxcode{\sphinxupquote{acoth  (x)}}
&
\sphinxAtStartPar
\(\coth^{-1}(x)\)
&
\sphinxAtStartPar
\sphinxcode{\sphinxupquote{atanh()}}
\\
\hline
\sphinxAtStartPar
\sphinxcode{\sphinxupquote{asin   (x)}}
&
\sphinxAtStartPar
\(\sin^{-1}(x)\)
&\\
\hline
\sphinxAtStartPar
\sphinxcode{\sphinxupquote{asinc  (x)}}
&
\sphinxAtStartPar
\(\frac{\sin^{-1}(x)}{x}\)
&\\
\hline
\sphinxAtStartPar
\sphinxcode{\sphinxupquote{asinh  (x)}}
&
\sphinxAtStartPar
\(\sinh^{-1}(x)\)
&
\sphinxAtStartPar
\sphinxcode{\sphinxupquote{asinh()}}
\\
\hline
\sphinxAtStartPar
\sphinxcode{\sphinxupquote{asinhc (x)}}
&
\sphinxAtStartPar
\(\frac{\sinh^{-1}(x)}{x}\)
&\\
\hline
\sphinxAtStartPar
\sphinxcode{\sphinxupquote{atan   (x)}}
&
\sphinxAtStartPar
\(\tan^{-1}(x)\)
&\\
\hline
\sphinxAtStartPar
\sphinxcode{\sphinxupquote{atan2  (x,y)}}
&
\sphinxAtStartPar
\(\tan^{-1}(\frac{x}{y})\)
&\\
\hline
\sphinxAtStartPar
\sphinxcode{\sphinxupquote{atanh  (x)}}
&
\sphinxAtStartPar
\(\tanh^{-1}(x)\)
&
\sphinxAtStartPar
\sphinxcode{\sphinxupquote{atanh()}}
\\
\hline
\sphinxAtStartPar
\sphinxcode{\sphinxupquote{ceil   (x)}}
&
\sphinxAtStartPar
\(\operatorname{ceil}(x)\)
&\\
\hline
\sphinxAtStartPar
\sphinxcode{\sphinxupquote{cos    (x)}}
&
\sphinxAtStartPar
\(\cos(x)\)
&\\
\hline
\sphinxAtStartPar
\sphinxcode{\sphinxupquote{cosh   (x)}}
&
\sphinxAtStartPar
\(\cosh(x)\)
&\\
\hline
\sphinxAtStartPar
\sphinxcode{\sphinxupquote{cot    (x)}}
&
\sphinxAtStartPar
\(\cot(x)\)
&\\
\hline
\sphinxAtStartPar
\sphinxcode{\sphinxupquote{coth   (x)}}
&
\sphinxAtStartPar
\(\coth(x)\)
&\\
\hline
\sphinxAtStartPar
\sphinxcode{\sphinxupquote{deg2rad(x)}}
&
\sphinxAtStartPar
\(\frac{\pi}{180} x\)
&\\
\hline
\sphinxAtStartPar
\sphinxcode{\sphinxupquote{exp    (x)}}
&
\sphinxAtStartPar
\(\exp(x)\)
&\\
\hline
\sphinxAtStartPar
\sphinxcode{\sphinxupquote{floor  (x)}}
&
\sphinxAtStartPar
\(\operatorname{floor}(x)\)
&\\
\hline
\sphinxAtStartPar
\sphinxcode{\sphinxupquote{frac   (x)}}
&
\sphinxAtStartPar
\(\operatorname{frac}(x)\)
&\\
\hline
\sphinxAtStartPar
\sphinxcode{\sphinxupquote{hypot  (x,y)}}
&
\sphinxAtStartPar
\(\sqrt{x^2+y^2}\)
&
\sphinxAtStartPar
\sphinxcode{\sphinxupquote{hypot()}}
\\
\hline
\sphinxAtStartPar
\sphinxcode{\sphinxupquote{hypot3 (x,y,z)}}
&
\sphinxAtStartPar
\(\sqrt{x^2+y^2+z^2}\)
&
\sphinxAtStartPar
\sphinxcode{\sphinxupquote{hypot()}}
\\
\hline
\sphinxAtStartPar
\sphinxcode{\sphinxupquote{invsqrt(x,v\_)}}
&
\sphinxAtStartPar
\(\frac{v}{\sqrt x}\)
&\\
\hline
\sphinxAtStartPar
\sphinxcode{\sphinxupquote{log    (x)}}
&
\sphinxAtStartPar
\(\log(x)\)
&\\
\hline
\sphinxAtStartPar
\sphinxcode{\sphinxupquote{log10  (x)}}
&
\sphinxAtStartPar
\(\operatorname{log10}(x)\)
&\\
\hline
\sphinxAtStartPar
\sphinxcode{\sphinxupquote{pow    (x,y)}}
&
\sphinxAtStartPar
\(x^y\)
&\\
\hline
\sphinxAtStartPar
\sphinxcode{\sphinxupquote{rad2deg(x)}}
&
\sphinxAtStartPar
\(\frac{180}{pi} x\)
&\\
\hline
\sphinxAtStartPar
\sphinxcode{\sphinxupquote{round  (x)}}
&
\sphinxAtStartPar
\(\operatorname{round}(x)\)
&
\sphinxAtStartPar
\sphinxcode{\sphinxupquote{round()}}
\\
\hline
\sphinxAtStartPar
\sphinxcode{\sphinxupquote{sign   (x)}}
&
\sphinxAtStartPar
\(-1, 0\text{ or }1\)
&
\sphinxAtStartPar
\sphinxcode{\sphinxupquote{mad\_num\_sign()}}
\\
\hline
\sphinxAtStartPar
\sphinxcode{\sphinxupquote{sign1  (x)}}
&
\sphinxAtStartPar
\(-1\text{ or }1\)
&
\sphinxAtStartPar
\sphinxcode{\sphinxupquote{mad\_num\_sign1()}}
\\
\hline
\sphinxAtStartPar
\sphinxcode{\sphinxupquote{sin    (x)}}
&
\sphinxAtStartPar
\(\sin(x)\)
&\\
\hline
\sphinxAtStartPar
\sphinxcode{\sphinxupquote{sinc   (x)}}
&
\sphinxAtStartPar
\(\frac{\sin(x)}{x}\)
&\\
\hline
\sphinxAtStartPar
\sphinxcode{\sphinxupquote{sinh   (x)}}
&
\sphinxAtStartPar
\(\sinh(x)\)
&\\
\hline
\sphinxAtStartPar
\sphinxcode{\sphinxupquote{sinhc  (x)}}
&
\sphinxAtStartPar
\(\frac{\sinh(x)}{x}\)
&\\
\hline
\sphinxAtStartPar
\sphinxcode{\sphinxupquote{sqrt   (x)}}
&
\sphinxAtStartPar
\(\sqrt{x}\)
&\\
\hline
\sphinxAtStartPar
\sphinxcode{\sphinxupquote{tan    (x)}}
&
\sphinxAtStartPar
\(\tan(x)\)
&\\
\hline
\sphinxAtStartPar
\sphinxcode{\sphinxupquote{tanh   (x)}}
&
\sphinxAtStartPar
\(\tanh(x)\)
&\\
\hline
\sphinxAtStartPar
\sphinxcode{\sphinxupquote{lgamma (x,tol)}}
&
\sphinxAtStartPar
\(\ln|\Gamma(x)|\)
&
\sphinxAtStartPar
\sphinxcode{\sphinxupquote{lgamma()}}
\\
\hline
\sphinxAtStartPar
\sphinxcode{\sphinxupquote{tgamma (x,tol)}}
&
\sphinxAtStartPar
\(\Gamma(x)\)
&
\sphinxAtStartPar
\sphinxcode{\sphinxupquote{tgamma()}}
\\
\hline
\sphinxAtStartPar
\sphinxcode{\sphinxupquote{trunc  (x)}}
&
\sphinxAtStartPar
\(\operatorname{trunc}(x)\)
&\\
\hline
\sphinxAtStartPar
\sphinxcode{\sphinxupquote{unit   (x)}}
&
\sphinxAtStartPar
\(\frac{x}{|x|}\)
&\\
\hline
\end{longtable}\sphinxatlongtableend\end{savenotes}


\section{Generic Functions (complex\sphinxhyphen{}like)}
\label{\detokenize{functions:generic-functions-complex-like}}
\sphinxAtStartPar
Complex\sphinxhyphen{}like generic functions forward the call to the method of the same name from the first argument when the later is not a \sphinxstyleemphasis{number}, otherwise it implements a real\sphinxhyphen{}like compatibility layer using the equivalent representation \(x+0i\).


\begin{savenotes}\sphinxattablestart
\centering
\begin{tabulary}{\linewidth}[t]{|T|T|}
\hline
\sphinxstyletheadfamily 
\sphinxAtStartPar
Functions
&\sphinxstyletheadfamily 
\sphinxAtStartPar
Return values
\\
\hline
\sphinxAtStartPar
\sphinxcode{\sphinxupquote{cabs (z)}}
&
\sphinxAtStartPar
\(|z|\)
\\
\hline
\sphinxAtStartPar
\sphinxcode{\sphinxupquote{carg (z)}}
&
\sphinxAtStartPar
\(\arg(z)\)
\\
\hline
\sphinxAtStartPar
\sphinxcode{\sphinxupquote{conj (z)}}
&
\sphinxAtStartPar
\(z^*\)
\\
\hline
\sphinxAtStartPar
\sphinxcode{\sphinxupquote{cplx (x,y)}}
&
\sphinxAtStartPar
\(x+i\,y\)
\\
\hline
\sphinxAtStartPar
\sphinxcode{\sphinxupquote{imag (z)}}
&
\sphinxAtStartPar
\(\Im(z)\)
\\
\hline
\sphinxAtStartPar
\sphinxcode{\sphinxupquote{polar(z)}}
&
\sphinxAtStartPar
\(|z|\,e^{i\arg(z)}\)
\\
\hline
\sphinxAtStartPar
\sphinxcode{\sphinxupquote{proj (z)}}
&
\sphinxAtStartPar
\(\operatorname{Proj}(z)\)
\\
\hline
\sphinxAtStartPar
\sphinxcode{\sphinxupquote{real (z)}}
&
\sphinxAtStartPar
\(\Re(z)\)
\\
\hline
\sphinxAtStartPar
\sphinxcode{\sphinxupquote{rect (z)}}
&
\sphinxAtStartPar
\(\Re(z)\cos(\Im(z))+i\,\Re(z)\sin(\Im(z))\)
\\
\hline
\sphinxAtStartPar
\sphinxcode{\sphinxupquote{reim (z)}}
&
\sphinxAtStartPar
\((\Re(z), \Im(z))\)
\\
\hline
\end{tabulary}
\par
\sphinxattableend\end{savenotes}


\section{Generic Functions (Error\sphinxhyphen{}like)}
\label{\detokenize{functions:generic-functions-error-like}}
\sphinxAtStartPar
Error\sphinxhyphen{}like generic functions forward the call to the method of the same name from the first argument when the later is not a \sphinxstyleemphasis{number}, otherwise it calls a C wrapper to corresponding function from the Faddeeva library from the MIT (see \sphinxcode{\sphinxupquote{mad\_num.c}}).


\begin{savenotes}\sphinxattablestart
\centering
\begin{tabulary}{\linewidth}[t]{|T|T|T|}
\hline
\sphinxstyletheadfamily 
\sphinxAtStartPar
Functions
&\sphinxstyletheadfamily 
\sphinxAtStartPar
C functions for reals
&\sphinxstyletheadfamily 
\sphinxAtStartPar
C functions for complex
\\
\hline
\sphinxAtStartPar
\sphinxcode{\sphinxupquote{erf  (x,tol)}}
&
\sphinxAtStartPar
\sphinxcode{\sphinxupquote{mad\_num\_erf}}
&
\sphinxAtStartPar
\sphinxcode{\sphinxupquote{mad\_cnum\_erf}}
\\
\hline
\sphinxAtStartPar
\sphinxcode{\sphinxupquote{erfc (x,tol)}}
&
\sphinxAtStartPar
\sphinxcode{\sphinxupquote{mad\_num\_erfc}}
&
\sphinxAtStartPar
\sphinxcode{\sphinxupquote{mad\_cnum\_erfc}}
\\
\hline
\sphinxAtStartPar
\sphinxcode{\sphinxupquote{erfi (x,tol)}}
&
\sphinxAtStartPar
\sphinxcode{\sphinxupquote{mad\_num\_erfi}}
&
\sphinxAtStartPar
\sphinxcode{\sphinxupquote{mad\_cnum\_erfi}}
\\
\hline
\sphinxAtStartPar
\sphinxcode{\sphinxupquote{erfcx(x,tol)}}
&
\sphinxAtStartPar
\sphinxcode{\sphinxupquote{mad\_num\_erfcx}}
&
\sphinxAtStartPar
\sphinxcode{\sphinxupquote{mad\_cnum\_erfcx}}
\\
\hline
\sphinxAtStartPar
\sphinxcode{\sphinxupquote{wf   (x,tol)}}
&
\sphinxAtStartPar
\sphinxcode{\sphinxupquote{mad\_num\_wf}}
&
\sphinxAtStartPar
\sphinxcode{\sphinxupquote{mad\_cnum\_wf}}
\\
\hline
\end{tabulary}
\par
\sphinxattableend\end{savenotes}


\section{Generic Functions (Length\sphinxhyphen{}Angle based)}
\label{\detokenize{functions:generic-functions-length-angle-based}}
\sphinxAtStartPar
Length\sphinxhyphen{}Angle base generic function relies on the following elementary relations between length and angle.
\begin{equation*}
\begin{split}l_{\text{arc}}  = a r = \frac{l_{\text{cord}}}{\operatorname{sinc}(\frac{a}{2})}
l_{\text{cord}} = 2 r \sin(\frac{a}{2}) = l_{\text{arc}} \operatorname{sinc}(\frac{a}{2})\end{split}
\end{equation*}

\begin{savenotes}\sphinxattablestart
\centering
\begin{tabulary}{\linewidth}[t]{|T|T|}
\hline
\sphinxstyletheadfamily 
\sphinxAtStartPar
Functions
&\sphinxstyletheadfamily 
\sphinxAtStartPar
Return values
\\
\hline
\sphinxAtStartPar
\sphinxcode{\sphinxupquote{arc2cord(l,a)}}
&
\sphinxAtStartPar
\(l \operatorname{sinc}(\frac{a}{2})\)
\\
\hline
\sphinxAtStartPar
\sphinxcode{\sphinxupquote{arc2len (l,a)}}
&
\sphinxAtStartPar
\(l \operatorname{sinc}(\frac{a}{2}) cos(a)\)
\\
\hline
\sphinxAtStartPar
\sphinxcode{\sphinxupquote{cord2arc(l,a)}}
&
\sphinxAtStartPar
\(\frac{l}{\operatorname{sinc}(\frac{a}{2})}\)
\\
\hline
\sphinxAtStartPar
\sphinxcode{\sphinxupquote{cord2len(l,a)}}
&
\sphinxAtStartPar
\(l cos(a)\)
\\
\hline
\sphinxAtStartPar
\sphinxcode{\sphinxupquote{len2arc (l,a)}}
&
\sphinxAtStartPar
\(\frac{l}{\operatorname{sinc}(\frac{a}{2}) cos(a)}\)
\\
\hline
\sphinxAtStartPar
\sphinxcode{\sphinxupquote{len2cord(l,a)}}
&
\sphinxAtStartPar
\(\frac{l}{cos(a)}\)
\\
\hline
\sphinxAtStartPar
\sphinxcode{\sphinxupquote{rangle  (a,r)}}
&
\sphinxAtStartPar
\(a + 2\pi \operatorname{round}(\frac{r-a}{2\pi})\)
\\
\hline
\end{tabulary}
\par
\sphinxattableend\end{savenotes}


\section{Generic Functions (Folding\sphinxhyphen{}Left based)}
\label{\detokenize{functions:generic-functions-folding-left-based}}

\begin{savenotes}\sphinxattablestart
\centering
\begin{tabulary}{\linewidth}[t]{|T|T|}
\hline
\sphinxstyletheadfamily 
\sphinxAtStartPar
Functions
&\sphinxstyletheadfamily 
\sphinxAtStartPar
Return values
\\
\hline
\sphinxAtStartPar
\sphinxcode{\sphinxupquote{sumsqr (x,y)}}
&
\sphinxAtStartPar
\(x^2 + y^2\)
\\
\hline
\sphinxAtStartPar
\sphinxcode{\sphinxupquote{sumabs (x,y)}}
&
\sphinxAtStartPar
\(|x| + |y|\)
\\
\hline
\sphinxAtStartPar
\sphinxcode{\sphinxupquote{minabs (x,y)}}
&
\sphinxAtStartPar
\(\min(|x|, |y|)\)
\\
\hline
\sphinxAtStartPar
\sphinxcode{\sphinxupquote{maxabs (x,y)}}
&
\sphinxAtStartPar
\(\max(|x|, |y|)\)
\\
\hline
\sphinxAtStartPar
\sphinxcode{\sphinxupquote{sumysqr(x,y)}}
&
\sphinxAtStartPar
\(x + y^2\)
\\
\hline
\sphinxAtStartPar
\sphinxcode{\sphinxupquote{sumyabs(x,y)}}
&
\sphinxAtStartPar
\(x + |y|\)
\\
\hline
\sphinxAtStartPar
\sphinxcode{\sphinxupquote{minyabs(x,y)}}
&
\sphinxAtStartPar
\(\min(x, |y|)\)
\\
\hline
\sphinxAtStartPar
\sphinxcode{\sphinxupquote{maxyabs(x,y)}}
&
\sphinxAtStartPar
\(\max(x, |y|)\)
\\
\hline
\end{tabulary}
\par
\sphinxattableend\end{savenotes}


\section{Non\sphinxhyphen{}Generic Functions}
\label{\detokenize{functions:non-generic-functions}}

\begin{savenotes}\sphinxattablestart
\centering
\begin{tabulary}{\linewidth}[t]{|T|T|}
\hline
\sphinxstyletheadfamily 
\sphinxAtStartPar
Functions
&\sphinxstyletheadfamily 
\sphinxAtStartPar
C or math functions
\\
\hline
\sphinxAtStartPar
\sphinxcode{\sphinxupquote{deg}}
&
\sphinxAtStartPar
\sphinxcode{\sphinxupquote{math.deg}}
\\
\hline
\sphinxAtStartPar
\sphinxcode{\sphinxupquote{fact}}
&
\sphinxAtStartPar
\sphinxcode{\sphinxupquote{mad\_num\_fact}}, \(n!\)
\\
\hline
\sphinxAtStartPar
\sphinxcode{\sphinxupquote{fmod}}
&
\sphinxAtStartPar
\sphinxcode{\sphinxupquote{math.fmod}}
\\
\hline
\sphinxAtStartPar
\sphinxcode{\sphinxupquote{frexp}}
&
\sphinxAtStartPar
\sphinxcode{\sphinxupquote{math.frexp}}
\\
\hline
\sphinxAtStartPar
\sphinxcode{\sphinxupquote{invfact}}
&
\sphinxAtStartPar
\sphinxcode{\sphinxupquote{mad\_num\_invfact}}, \(1/n!\)
\\
\hline
\sphinxAtStartPar
\sphinxcode{\sphinxupquote{ldexp}}
&
\sphinxAtStartPar
\sphinxcode{\sphinxupquote{math.ldexp}}
\\
\hline
\sphinxAtStartPar
\sphinxcode{\sphinxupquote{max}}
&
\sphinxAtStartPar
\sphinxcode{\sphinxupquote{math.max}}
\\
\hline
\sphinxAtStartPar
\sphinxcode{\sphinxupquote{min}}
&
\sphinxAtStartPar
\sphinxcode{\sphinxupquote{math.min}}
\\
\hline
\sphinxAtStartPar
\sphinxcode{\sphinxupquote{modf}}
&
\sphinxAtStartPar
\sphinxcode{\sphinxupquote{math.modf}}
\\
\hline
\sphinxAtStartPar
\sphinxcode{\sphinxupquote{rad}}
&
\sphinxAtStartPar
\sphinxcode{\sphinxupquote{math.rad}}
\\
\hline
\end{tabulary}
\par
\sphinxattableend\end{savenotes}


\section{Random number generators}
\label{\detokenize{functions:random-number-generators}}

\chapter{Indices and tables}
\label{\detokenize{index:indices-and-tables}}\begin{itemize}
\item {} 
\sphinxAtStartPar
\DUrole{xref,std,std-ref}{genindex}

\item {} 
\sphinxAtStartPar
\DUrole{xref,std,std-ref}{modindex}

\item {} 
\sphinxAtStartPar
\DUrole{xref,std,std-ref}{search}

\end{itemize}



\renewcommand{\indexname}{Index}
\printindex
\end{document}